\documentclass[t]{beamer}

\usetheme{CambridgeUS}
\usecolortheme{beaver}
\setbeamertemplate{navigation symbols}{}

\usepackage[utf8]{inputenc}
\usepackage[croatian]{babel}

\usepackage{datetime}
\renewcommand{\dateseparator}{.}
\newcommand{\todayiso}{\twodigit\day \dateseparator \twodigit\month \dateseparator \the \year}
\date{\todayiso}

\usepackage{listing}
\usepackage{graphicx}
\usepackage{subcaption}
\captionsetup{compatibility=false}


\title[NKOSL]{Napredno korištenje operacijskog sustava Linux}
\author[Leonard Volarić Horvat]{Leonard Volarić Horvat\\{\small Nositelj: Stjepan Groš}}
\subtitle{1. Boot proces i datotečni sustavi}
\institute[FER]{Sveučilište u Zagrebu\\Fakultet elektrotehnike i računarstva}

\begin{document}

{
	\setbeamertemplate{footline}{}
	\begin{frame}
		\maketitle
	\end{frame}
}

\begin{frame}
	\frametitle{Sadržaj}
	\tableofcontents
\end{frame}

\section{Boot}
\begin{frame}
    \frametitle{Od uključivanja do /bin/bash}
    \begin{itemize}
        \item BIOS/UEFI
        \item MBR/GPT
        \item Bootloader i odabir OS-a
        \item Učitavanje initramdiska i kernela
        \item Init proces 
        \item Bash
    \end{itemize}
\end{frame}




\section{BIOS/UEFI}
\begin{frame}
	\frametitle{BIOS}
	\textbf{B}asic \textbf{I}nput and \textbf{O}utput \textbf{S}ystem
	\begin{itemize}
		\item Firmware na matičnoj ploči - spremljen u ROM čipu
		\item Prvi softver koji se pokreće
		\item Izvršava Power-On Self-Test (POST) nad matičnom pločom i u nju  spojenim uređajima (S.M.A.R.T. test diskova, Extension ROM dodatnih uređaja itd.)
		\item Konfigurira se preko ugrađenog firmwarea ili hardverski (jumperi, DIP switchevi, ... )
		\item Omogućuje odabir uređaja s kojeg se pokreće operacijski sustav (\textit{boot sequence})
	\end{itemize}
\end{frame}

\begin{frame}
	\frametitle{UEFI}
	\textbf{U}nified \textbf{E}xtensible \textbf{F}irmware \textbf{I}nterface
	\begin{itemize}
		\item Specifikacija sučelja između OS-a i \textit{firmwarea}
		\item Nadgradnja BIOS-a
		\item Nudi sve funkcionalnosti BIOS-a
		\item Puno naprednije mogućnosti
		\begin{itemize}
			\item Servisi, aplikacije, driveri ...
			\item EFI system partition (ESP)
		\end{itemize}
		\item Najveći doprinos specifikacije - \textbf{GPT}, nasljednik MBR-a
	\end{itemize}
\end{frame}

\begin{frame}
	\frametitle{UEFI}
	\begin{itemize}
		\item Boot manager je dio UEFI specifikacije
		\item Traži i izvršava bootloader s neke od particija
		\item Detekcija UEFI bootloadera na sustavu
	\end{itemize}
	\begin{itemize}
		\item \textit{Compatibility support module (CSM)}
		\begin{itemize}
			\item Omogućuje \textit{backward compatibility} s klasičnim bootloaderima
		\end{itemize}
	\end{itemize}
\end{frame}




\section{Particije; MBR/GPT}
\begin{frame}
	\frametitle{Particije}
	\begin{itemize}
		\item Hard disk sadrži particije formatirane određenim datotečnim sustavima
		\item Particije se koriste za funkcijsko i logičko odjeljivanje podataka
		\item Za pokretanje sustava s diska nužno je na njemu imati boot particiju
		\item Informacije o particijama sadržane su u
		\begin{itemize}
			\item \textbf{M}aster \textbf{B}oot \textbf{R}ecord (MBR), ili
			\item \textbf{G}UID \textbf{P}artition \textbf{T}able (GPT)
		\end{itemize}
	\end{itemize}
\end{frame}	





\begin{frame}
	\frametitle{MBR}
	\begin{itemize}
		\item Zapisan u prvom sektoru diska (512 B / 2 KiB)
		\item Sadrži redom:
		\begin{itemize}
			\item \textit{\textbf{Bootstrap code}} - minimalni podaci potrebni za nastavak boot procesa (446B)
			\item \textit{Partition table} - informacije o particijama (4*16B)
			\item \textit{Magic number} - zaštitna suma (2B)
		\end{itemize}
		\item Može adresirati najviše 2 TB:
		\begin{itemize}
			\item 4B (32 bita) za adresiranje sektora
			\item Sektori veličine 512B (2\textasciicircum{9} B)
			\item $\longrightarrow{2}^{32}*{2}^{9}B={2}^{41}B=2*{2}^{40}B=$2 TB
		\end{itemize}
	\end{itemize}
\end{frame}


\begin{frame}
	\begin{itemize}
		\item Podržava (samo) 4 \emph{primarne} particije
		\item Za veći broj particija koriste se \emph{logičke} particije koje se upisuju u jednu primarnu particiju tipa \emph{extended}
		\item U pravilu se MBR koristi s BIOS-om
		\begin{itemize}
			\item UEFI-eva kompatibilnost unatrag omogućuje korištenje MBR-a i s UEFI-em
			\item Nije idealno i ne koristi sve mogućnosti UEFI specifikacije
		\end{itemize}
	\end{itemize}
\end{frame}

\begin{frame}
	\frametitle{GPT}
	\begin{itemize}
		\item Služi istoj svrsi kao i MBR, ali je napredniji i nudi više mogućnosti
		\begin{itemize}
			\item Može adresirati 9.4 zetabajta (9.4*10\textasciicircum9 TB)
			\item Nema realnog ograničenja na broj particija
			\item Replikacija GPT-a na više mjesta na disku $\longrightarrow$ robusnost
		\end{itemize}
		\item Sadrži i "protective MBR" na nultoj logičkoj adresi
		\begin{itemize}
			\item Zaštita od "nesporazuma" s alatima koji očekuju MBR disk
			\item \textit{backward compatibility} s BIOS-om
		\end{itemize}
		\item Za bootanje se koristi isključivo s UEFI-em
		\begin{itemize}
			\item BIOS može koristiti GPT diskove za podatke - ali samo do 2TB!
		\end{itemize}
	\end{itemize}
\end{frame}





\section{Bootloader}
\begin{frame}
	\frametitle{Bootloader}
	\begin{itemize}
		\item Izvršava se nakon POST-a
		\item Pokreće operacijski sustav - učitava kernel
	\end{itemize}
	
	\begin{itemize}
		\item Prva faza bootloadera zapisana u tzv. \textit{bootstrap code} MBR diskova
		\item Često se pokreće druga faza (npr. GRUB) s neke od particija
	\end{itemize}
	\begin{itemize}
		\item UEFI\&GPT diskovi imaju suštinski sličan, ali drugačije implementiran mehanizam
		\item EFI System Partition sadrži:
		\begin{itemize}
	 		\item Bootloadere
	 		\item Informacije o kernelima instaliranih OS-eva
	 		\item Informacije o driverima
	 		\item \dots
		\end{itemize}
	\end{itemize}
\end{frame}

\begin{frame}
	\frametitle{Popularni bootloaderi}
	\textbf{GR}and \textbf{U}nified \textbf{B}ootloader
	\begin{itemize}
		\item GNU projekt
		\item Konfigurabilan za više operacijskih sustava na jednom računalu (\textit{multiboot})
	\end{itemize}
	\begin{itemize}
		\item \textbf{GRUB Legacy}
		\begin{itemize}
			\item Starija verzija
			\item Više se ne razvija
		\end{itemize}
		\item \textbf{GRUB 2}
		\begin{itemize}
			\item Drugačije konfiguracijske datoteke
			\item Modularan, puno mogućnosti
			\item Zauzima puno više prostora od GRUB Legacy
		\end{itemize}
	\end{itemize}
	\vfill
	\textbf{Syslinux}
	\begin{itemize}
		\item \emph{Lightweight} bootloader
		\item Jednostavna konfiguracija
	\end{itemize}
\end{frame}



\section{initrd, kernel}
\begin{frame}
	\frametitle{initrd/initramfs}
	\begin{itemize}
		\item Kernel prepoznaje sve uređaje na sustavu i učitava \textbf{init}ial \textbf{r}am\textbf{d}isk
		\item Riječ je o pomoćnom datotečnom sustavu učitanom direktno u RAM
		\item Omogućuje kernelu učitavanje osnovnih modula kako bi došao do \emph{root} filesystema
		\begin{itemize}
			\item moduli
			\item enkripcija
			\item RAID / LVM
			\item \dots
		\end{itemize}
		\item initrd se po završetku briše iz memorije
		\item Pokreće se init
	\end{itemize}
\end{frame}


\begin{frame}
	\frametitle{Zašto initrd?}
	\begin{itemize}
		\item initrd sadrži osnovni i vrlo mali \textit{filesystem} koji zna baratati diskom na kojem se nalazi kernel
		\item Nužan za učitavanje (inherentno većeg) kernela
	\end{itemize}
	\begin{itemize}
		\item Važan dio koncepcije otvorenih računarskih sustava je \textbf{modularnost}
		\item Od otvorenog se sustava očekuje podrška za lagano proširivanje funkcionalnosti kroz \textbf{module}
		\item initrd upravlja učitavanjem samo potrebnih modula pri pokretanju sustava
		\item Ostali se moduli mogu uključivati po potrebi (korištenjem naredbe \textit{modprobe})
	\end{itemize}
\end{frame}



\section{init, systemd}
\begin{frame}
	\frametitle{init}
	\begin{itemize}
		\item init je sustav zadužen za inicijalizaciju OS-a, a pokreće se odmah nakon pokretanja kernela
		\item System V (jedna od prvih komercijalnih Unix distribucija) uvodi koncept \emph{runlevela}
		\item runlevel je način rada sustava
	\end{itemize}
	\begin{itemize}
		\item U Linuxu definirano nekoliko runlevela
		\begin{description}
			\item[0] - Sustav se isključuje
			\item[1 / S] - Single-user mode
			\item[2 - 5] - Multi-user mode, nekoliko inačica
			\item[6] - Ponovno pokretanje
		\end{description}
		\item Stvarni \textit{startup} level je određen u datoteci \texttt{/etc/inittab}
	\end{itemize}
\end{frame}




\begin{frame}
	\frametitle{init}
	\begin{itemize}
		\item Kod ulaska u svaki runlevel pokreću se skripte
		\item[] \texttt{/etc/rc.d/rc?.d}
		\item Poredak izvođenja skripti je određen imenom skripte u direktoriju:
		\begin{itemize}
			\item[] \textbf{K01-K99} - skripta gasi servis pri ulasku u runlevel
			\item[] \textbf{S01-S99} - skripta pokreće servis pri ulasku u runlevel
		\end{itemize}
		\item Nakon izvođenja svih skripti u direktoriju pokreće se i \texttt{/etc/rc.local}
		\item Sve što se nalazi u direktorijima \texttt{/etc/rc.d} je obično samo symlink na skripte u \texttt{/etc/init.d}

	\end{itemize}
\end{frame}




\begin{frame}
	\frametitle{systemd}
	\begin{itemize}
		\item Moderni sustav upravljanja procesima i cijelim sustavom
		\item Istiskuje iz uporabe monolitni i glomazni init 
		\item systemd koristi svaka popularna moderna distribucija
	\end{itemize}
	\begin{itemize}
		\item Konfiguracija zasnovana na \emph{unit} datotekama
		\item Neki tipovi \emph{unit} datoteka
		\begin{description}
			\item[daemon] označava pozadinske, neinteraktivne procese (provjeravanje IO uređaja, logovi, mreža...)
			\item[target] grupira unite u skupine
			\item[service, socket, ...]
		\end{description}
	\end{itemize}
\end{frame}

\begin{frame}
	\frametitle{systemd}
	\begin{itemize}
		\item Upravljanje systemd sustavom: naredba \textit{systemctl}
			\begin{itemize}
				\item \textit{systemctl} - ispis svih jedinica
				\item \textit{systemctl status ime} - ispis trenutnog stanja jedinice \textit{ime}
				\item \textit{systemctl enable httpd} - pokretanje httpd daemona pri pokretanju sustava
				\item \textit{systemctl start httpd} - instantno pokretanje httpd daemona
			\end{itemize}
		\item Upravljanje logovima systemd sustava: naredba \textit{journalctl}
			\begin{itemize}
				\item systemd logove ne zapisuje tekstualno nego u binarnom zapisu
				\item journalctl omogućuje čitanje tih zapisa
			\end{itemize}
		\item Analiza vremena trajanja systemd sustava pri pokretanju OS-a: naredba \textit{systemd-analyze}
			\begin{itemize}
				\item \textit{systemd-analyze blame} - distribucija trajanja po procesima pri pokretanju
			\end{itemize}
		
	\end{itemize}
\end{frame}

\begin{frame}
	\frametitle{Primjer dodavanja systemd servirsa}
	\begin{itemize}
		\item Servisi se dodaju u direktorij /etc/systemd/system/
		\item Primjer: /etc/systemd/system/moj-program.service
		\item Unit fajl opisuje servis i sastoji se od nekoliko glavnih dijelova:
		\begin{itemize}
			\item [Unit] - opis jedinice i ovisnosti o ostalim jedinicama
			\item [Service] - implementacijski detalji servisa - tip servisa, korisnik, okolina, program za pokretanje...
			\item [Install] 
		\end{itemize}
		\item Na sljedećem je slajdu primjer systemd unit fajla koji opisuje servis
		\item Nakon dodavanja user fajla potrebno je pokrenuti naredbu \textit{systemctl daemon-reload}
		\item Nakon toga je za pokretanje servisa dovoljno pokrenuti \textit{systemctl start moj\_program}
	\end{itemize}
\end{frame}

\begin{frame}[fragile]
	\footnotesize
	\begin{verbatim}
			[Unit]
			Description=The Nice and Accurate Prophecies
			After=network.target # ne pokreći prije dizanja mreže
	\end{verbatim}
	\begin{verbatim}
			[Service]
			Type=simple
			User=my_username		
			ExecStart=/opt/my_program.sh
			# tu se mogu dodati stvari poput
			# ExecStop=/opt/program.sh --stop # ako postoji 
			# neki poseban način za zaustaviti program
			# ExecReload=/opt/program.sh --reload # ako postoji 
			# neki poseban način za restartati program
			# systemd ima defaultan način za gašenje programa (signali) 
			# tako da nije potrebno eksplicitno pisati to
			# i po defaultu restart smatra gašenjem i paljenjem
	\end{verbatim}
	\begin{verbatim}
			# ovdje mogu doći fajlovi koji specificiraju environment varijable
			Restart=always
			# systemd sada prati je li program živ i restarta ga ako nije
	\end{verbatim}
	\begin{verbatim}
			[Install]
			WantedBy=default.target # u kojoj grupi se pokreće servis
	\end{verbatim}
\end{frame}



\section{Kernel}
\begin{frame}[fragile]
	\frametitle{Kernel moduli}
	\textit{"Kernel modules are pieces of code that can be loaded and unloaded into the kernel upon demand. They extend the functionality of the kernel without the need to reboot the system."}
	\begin{itemize}
		\item Moduli su spremljeni na lokaciji
		\item[] \verb|/usr/lib/modules/kernel_release|
	\end{itemize}
	\begin{itemize}
		\item \texttt{lsmod} - trenutno učitani moduli
		\item \texttt{modprobe module\_name parameter\_name=parameter\_value} - učitavanje modula
		\item \texttt{modprobe -r module\_name} - \textit{unload} modula
	\end{itemize}
	\begin{itemize}
		\item[] \verb|/etc/modprobe.d/|
		\item[] \verb|  options module_name parameter_name=parameter_value|
	\end{itemize}
\end{frame}



\begin{frame}[fragile]
	\frametitle{Kernel domesticus}
	\framesubtitle{Kompajliranje kernela (CentOS)}
    \footnotesize
    \begin{verbatim}
	yum -y groupinstall 'Development tools'
	yum -y install ncurses-devel rpm-build qt-devel
	\end{verbatim}
	\begin{verbatim}
	cd /usr/src/kernels
	wget https://www.kernel.org/pub/linux/kernel/v3.x/linux-3.12.6.tar.bz2
	tar xvjf linux-3.12.6.tar.bz2
	cd /usr/src/kernels/linux-3.12.6
	\end{verbatim}
	Izmijeniti konfiguraciju (promjeniti verziju u Makefile):
	\begin{verbatim}
	cp /boot/config`-uname -r` .config
	make menuconfig \# ili make xconfig
	\end{verbatim}
	Spremiti .config i kopajlirati
	\begin{verbatim}
	make; make modules_install; make install;
	vi /boot/grub/grub.conf \# (default=1 to default=0)
	\end{verbatim}
\end{frame}

\begin{frame}
	\begin{itemize}
		\item Dakle, pokretanje sustava ukratko izgleda ovako:
		\begin{itemize}
			\item Procesor iz firmwarea čita i pokreće BIOS/UEFI
			\item BIOS/UEFI saznaju lokaciju bootloadera na disku iz MBR-a/GPT-a
			\item S diska se pokreće bootloader
			\item Bootloader pokreće odabrani OS
			\item OS prvo učitava initrd/initramfs da može učitati ostatak kernela
			\item Kada se učita kernel, on pokreće init proces s PID-om 1 (nekad SysVinit, danas uglavnom systemd)
			\item init proces pokreće sve potrebne pozadinske procese
			\item Korisnik pali terminal i u svojem omiljenom \textit{shellu} pokreće naredbu \textit{sl} i zadovoljan je
		\end{itemize}
	\end{itemize}
\end{frame}



\section*{}
\begin{frame}
	\frametitle{Literatura}
	\url{https://www.happyassassin.net/2014/01/25/uefi-boot-how-does-that-actually-work-then/} \\
	\vfill
	\url{https://wiki.archlinux.org/index.php/GRUB} \\
	\url{http://wiki.gentoo.org/wiki/GRUB2_Quick_Start} \\
	\url{https://wiki.archlinux.org/index.php/Syslinux}
	\vfill
	\url{https://wiki.archlinux.org/index.php/Kernel_parameters} \\
	\url{https://wiki.archlinux.org/index.php/Unified_Extensible_Firmware_Interface}
	\vfill
	\url{http://0pointer.de/blog/projects/the-biggest-myths.html}
	\vfill
	\url{http://users.cecs.anu.edu.au/~okeefe/p2b/power2bash/power2bash.html}\\
\end{frame}

\end{document}
