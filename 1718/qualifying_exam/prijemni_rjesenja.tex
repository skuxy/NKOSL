\documentclass[a4paper,11pt]{exam}
\usepackage[left=1.5cm, right=1.5cm, top=3cm]{geometry}
\usepackage[utf8]{inputenc}
\usepackage[T1]{fontenc}
\setlength\parindent{0pt}
\renewcommand{\familydefault}{\sfdefault}
\newcommand{\shell}[1]{\texttt{#1}}

\CorrectChoiceEmphasis{\bfseries\itshape}
\printanswers

\usepackage{xpatch}
\xpatchcmd{\oneparchoices}{\penalty -50\hskip 1em plus 1em\relax}{\hfill}{}{}
\xpatchcmd{\oneparchoices}{\penalty -50\hskip 1em plus 1em\relax}{\hfill}{}{}

\begin{document}
\firstpageheader
	{}
	{{\large \textbf{Napredno korištenje OS Linux}}\\
		\textbf{Prijemni ispit - 7. ožujka 2018.}}
	{}
\footer
	{}{}{\thepage}
	
Ime i prezime: \fillin[][7cm] \hfill JMBAG: \fillin[][5cm]\\

\begin{questions}
	\question
	Najveći autoritet u današnjem razvoju Linuxa ima:
	
	\begin{oneparchoices}
		\CorrectChoice Linus Torvalds
		\choice Richard Stallman
		\choice Lennart Poettering
		\choice Steve Ballmer
	\end{oneparchoices}
	
	\question
	Debian, Fedora i OpenSUSE samo su neke od Linux
	
	\begin{oneparchoices}
		\CorrectChoice distribucija
		\choice klasa
		\choice komponenata
		\choice segmenata
	\end{oneparchoices}
	
	\question
	Najraširenija licenca u razvoju Linuxa općenito jest:
	
	\begin{oneparchoices}
		\choice BSD
		\choice MIT
		\CorrectChoice GPL
		\choice Apache
	\end{oneparchoices}
	
	\question
  Dva načina distribucije ažuriranja su: 

	\begin{oneparchoices}
		    \choice stable i testing
        \CorrectChoice rolling i stable 
        \choice rolling i rocking
        \choice unstable i stable
	\end{oneparchoices}

	\question
  Direktorij u koji korisnik ručno dodaje pokretne uređjaje/medije jest:
	
	\begin{oneparchoices}
        \choice /bin
        \CorrectChoice /mnt
        \choice /lib
        \choice /etc
	\end{oneparchoices}
	
	\question
Zauzece prostora po particija provjeravamo naredbom:
	
	\begin{oneparchoices}
		\choice ds 
		\choice ls 
		\CorrectChoice df 
		\choice free
	\end{oneparchoices}
	
	\question
  Prazna datoteka stvara se naredbom:

	\begin{oneparchoices}
		\choice mk
		\CorrectChoice touch
		\choice mv
		\choice pwd
	\end{oneparchoices}
	
	\question
	Kojom od sljedećih naredbi stvaramo direktorij "linux" i u njemu direktorij "osnovni"? 
	
	\begin{oneparchoices}
		\choice \shell{touch -p linux/osnovni} \\
		\choice \shell{touch linux/osnovni} \\
		\choice \shell{mkdir linux/osnovni} \\
		\CorrectChoice \shell{mkdir -p linux/osnovni}
	\end{oneparchoices}
	
	\question
  Nalazimo se u direktoriju \shell{/home/pedro/.config/hexchat/assets}. U kojem se direktoriju nalazimo nakon izvođenja naredbe  \shell{cd ../../././../} ? 

	\begin{oneparchoices}
		\choice \shell{/home/pedro/.config/hexchat} \\
		\choice \shell{/home/} \\
		\choice \shell{/home/pedro/.config/} \\
		\CorrectChoice \shell{/home/pedro/}
	\end{oneparchoices}
	
	\question
 Znakovi kojima se vrši preusmjeravanje stdout i stderr jesu: 
	\begin{oneparchoices}
		\choice 2>\&1 
		\CorrectChoice \&> 
		\choice >
		\choice >2
	\end{oneparchoices}
	
	\question
Yes je naredba koja će u slučaju pozivanja bez argumenata na stdout ispisivati ‘y’. Ponekad skriptama ne želimo ručno odgovarati ‘y’ i želimo da se sve izvede kao da smo na svako pitanje odgovorili ‘y’. Dobar primjer je naredba \shell{rm -ri <direktorij>}. Koja naredba će nam omogućiti takvo izvođenje:
	
	\begin{oneparchoices}
		\choice \shell{rm -ri | yes} 
		\CorrectChoice \shell{yes | rm -ri} 
		\choice \shell{rm -ri \&\& yes} 
		\choice \shell{yes || rm -ri}
	\end{oneparchoices}
	
	\question
  Da bismo preusmjerili \textbf{samo} ispis pogreški neke naredbe u datoteku, na kraj naredbe dodajemo:

	\begin{oneparchoices}
	  \choice \shell{1>./datoteka} 
		\CorrectChoice \shell{2>./datoteka} 
		\choice \shell{\&>./datoteka} 
		\choice \shell{>./datoteka}
	\end{oneparchoices}
	
	\question
	Izbaci uljeza!
	
	\begin{oneparchoices}
		\CorrectChoice before
		\choice while
		\choice until
		\choice for
	\end{oneparchoices}

	\question
	Naredba ls -ld /etc/??? izlistava:
	\begin{oneparchoices}
		\choice sve datoteke koje čiji naziv ima najmanje tri znaka
		\choice sve direktorije čiji naziv počinje s ???
    \CorrectChoice sve direktorije čiji naziv ima točno tri znaka
	  \choice sve datoteke s nazivom ???
	\end{oneparchoices}
	
	\question
  U skriptnom jeziku bash, svaka varijabla je tipa:
	
	\begin{oneparchoices}
		\choice integer
		\CorrectChoice string
		\choice double
		\choice char
	\end{oneparchoices}
	
	\question
  Kojim znakom oznacavamo kraj reda u regularnim izrazima?

	\begin{oneparchoices}
		\choice \textasciicircum
		\CorrectChoice \$
		\choice >
		\choice *
	\end{oneparchoices}

	\question
	Rezultat izvođenja naredbe \verb| sed -r "s/(.*)\.(.*)\.(.*)/\2\1\3/"| nad nizom A.B.C jest:
	
	\begin{oneparchoices}
		\choice CBA
		\choice ABC
		\CorrectChoice BAC
		\choice CAB
	\end{oneparchoices}
	
	\question
	Odaberite naredbu kojom će se iz niza koji je sastavljen od proizvoljnog broja znamenki nakon kojih slijedi riječ (npr. 1337banana, 90000mango) maknuti niz brojeva i sva slova prebaciti u velika:

	\begin{oneparchoices}
		\CorrectChoice \verb|sed -r "s/([[:digit:]])*(.*)/\U\2/"| \\ 
		\choice \verb|sed -r "s/([[:digit:]])(.*)/\u\2/"|  \\
		\choice \verb|sed -r "s/([[:alnum:]])*(.*)/\u\1/"| \\
		\choice \verb|sed -r "s/([[:alnum:]])(.*)/\U\2"| 
	\end{oneparchoices}


  \question
  Naredba kojom se briše niz .? na kraju retka jest: 

		\begin{oneparchoices}
		\choice \verb|sed -r "s/(.*)(.?)\$/\1/"|  \\
		\choice \verb|sed -r "s/\.*.?\$/\1/"| \\
    \CorrectChoice \verb|sed -r "s/(.*)(\.\?)\$/\1/"| \\ 
    \choice \verb|sed -r "s/(\.*)(\.\?)\$/\2/"|
	\end{oneparchoices}

	
		\question
	Redak \shell{root:T3RqrzxU1MAH3F3wtuQu/:13284:0:99999:7:::} primjer je retka iz datoteke:
	
	\begin{oneparchoices}
		\choice \shell{/etc/root} 
		\choice \shell{/etc/sudoers} 
		\choice \shell{/etc/fstab} 
		\CorrectChoice \shell{/etc/shadow}
  \end{oneparchoices}

		\question
	Redak \shell{GROUPTWO    ALL = NOPASSWD: /usr/bin/updatedb, PASSWD: /bin/kill} primjer je retka iz datoteke:
	
	\begin{oneparchoices}
		\choice \shell{/etc/root} 
		\CorrectChoice \shell{/etc/sudoers} 
		\choice \shell{/etc/fstab} 
		\choice \shell{/etc/shadow}
  \end{oneparchoices}


	\question
	Primarna grupa svakog korisnika zapisana je u datoteci:

	\begin{oneparchoices}
   	\choice \shell{/etc/hosts}
		\CorrectChoice \shell{/etc/passwd}
		\choice \shell{/usr/passwd} 
		\choice \shell{/usr/group}  
	\end{oneparchoices}

	\question
  Naredba \shell{su} bez argumenata:
	
	\begin{oneparchoices}
		\choice daje trenutnom korisniku sudo prava \\
		\CorrectChoice mijenja trenutnog korisnika u \shell{root} korisnika \\ 
		\choice stvara novog usera \\
		\choice ispisuje pogrešku zbog nepravilnog poziva
	\end{oneparchoices}
	
	\question
  Naredba kojom omogućujemo samo čitanje određene datoteke jest: 

	\begin{oneparchoices}
		\CorrectChoice \shell{chmod 444 datoteka}\\ 
		\choice \shell{chmod 755 datoteka} \\
		\choice \shell{chmod 777 datoteka} \\
		\choice \shell{chmod 333 datoteka}
	\end{oneparchoices}
	
	\question
	Naredba koja dozvoljava brisanje datoteka unutar direktorija samo vlasniku i rootu 
	
	\begin{oneparchoices}
		\choice \shell{chmod +X direktorij} \\
		\choice \shell{chmod +r direktorij} \\
    	\CorrectChoice \shell{chmod +t direktorij}\\ 	
	  \choice \shell{chmod -X direktorij}
	\end{oneparchoices}

	\question
	Koju sintaksu biste upotrijebili za pokretanje procesa u pozadini?

	\begin{oneparchoices}
		\choice \shell{<naredba> \$} \\
		\CorrectChoice \shell{<naredba> \&} \\
		\choice \shell{<naredba> !} \\
		\choice \shell{<naredba> \textasciicircum} \\
	\end{oneparchoices}
	
	\question
  Što sadrži varijabla \shell{\$\$}?

	\begin{oneparchoices}
				\choice trenutnu verziju kernela\\
		\CorrectChoice \textit{PID} trenutnog procesa \\ 
		\choice \textit{exit code} posljednjeg pokrenutog programa\\
		\choice podrazumijevani emulator terminala
	\end{oneparchoices}
	
	\question
  Koji se signal šalje sa pritiskom kombinacije tipki Ctrl+C? 

  \begin{oneparchoices}
		\choice \shell{SIGKILL}
		\choice \shell{SIGTRAP}
    \choice \shell{SIGHUP}
    \CorrectChoice \shell{SIGINT}  
  \end{oneparchoices}

	\question
  Odaberite "najjači" signal:

  \begin{oneparchoices}
		\CorrectChoice \shell{SIGKILL}
		\choice \shell{SIGTRAP}
    \choice \shell{SIGHUP}
    \choice \shell{SIGINT}  
  \end{oneparchoices}


	\question
Pomoću koje kombinacije tipki je moguće suspendirati privremeno zaustaviti aktivni proces u ljusci?

  \begin{oneparchoices}
		\choice \shell{CTRL+C}
		\CorrectChoice \shell{CTRL+Z}
    \choice \shell{ALT+C}
    \choice \shell{CTRL+D}  
  \end{oneparchoices}
	
    
	\vspace{1em}
\uplevel{
	Preferencija termina:
	\\ A. Jutarnji (10-12) \hspace{20mm} B. Popodnevni (12-14)
}
	%\begin{oneparchoices}
	%	\choice Jutarnji (10-12) \\
	%	\choice Popodnevni (12-14)
	%\end{oneparchoices}

\end{questions}
\end{document}
