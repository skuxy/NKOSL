\documentclass[a4paper,11pt]{exam}
\usepackage[left=1.5cm, right=1.5cm, top=3cm]{geometry}
\usepackage[utf8]{inputenc}
\usepackage[T1]{fontenc}
\setlength\parindent{0pt}
\renewcommand{\familydefault}{\sfdefault}
\newcommand{\shell}[1]{\texttt{#1}}

\CorrectChoiceEmphasis{\bfseries\itshape}
%\printanswers

\usepackage{xpatch}
\xpatchcmd{\oneparchoices}{\penalty -50\hskip 1em plus 1em\relax}{\hfill}{}{}
\xpatchcmd{\oneparchoices}{\penalty -50\hskip 1em plus 1em\relax}{\hfill}{}{}

\begin{document}
\firstpageheader
	{}
	{{\large \textbf{Advanced Use of Linux Operating System}}\\
		\textbf{Qualifying Exam - March 7th, 2018}}
	{}
\footer
	{}{}{\thepage}
	
Name: \fillin[][7cm] \hfill ID Number: \fillin[][5cm]\\

\begin{questions}
	\question
	Who holds the leading authority in Linux development today?
	
	\begin{oneparchoices}
		\CorrectChoice Linus Torvalds
		\choice Richard Stallman
		\choice Lennart Poettering
		\choice Steve Ballmer
	\end{oneparchoices}
	
	\question
	Debian, Fedora and OpenSUSE are examples of many Linux
	
	\begin{oneparchoices}
		\CorrectChoice distributions
		\choice classes
		\choice components
		\choice segments
	\end{oneparchoices}
	
	\question
	The licence most commonly used with Linux is:
	
	\begin{oneparchoices}
		\choice BSD
		\choice MIT
		\CorrectChoice GPL
		\choice Apache
	\end{oneparchoices}
	
	\question
  Two types of distribution updates are: 

	\begin{oneparchoices}
		    \choice stable and testing
        \CorrectChoice rolling and stable 
        \choice rolling and rocking
        \choice unstable and stable
	\end{oneparchoices}

	\question
  Which of these is commonly used to manually add removable storage?
	
	\begin{oneparchoices}
        \choice /bin
        \CorrectChoice /mnt
        \choice /lib
        \choice /etc
	\end{oneparchoices}
	
	\question
One can check disk space usage by running which of these commands?
	
	\begin{oneparchoices}
		\choice ds 
		\choice ls 
		\CorrectChoice df 
		\choice free
	\end{oneparchoices}
	
	\question
  An empty file can be created by running:

	\begin{oneparchoices}
		\choice mk
		\CorrectChoice touch
		\choice mv
		\choice pwd
	\end{oneparchoices}
	
	\question
	Which of the following commands would create a directory called "\shell{linux}", and another one within it, called "\shell{osnovni}"? 
	
	\begin{oneparchoices}
		\choice \shell{touch -p linux/osnovni} \\
		\choice \shell{touch linux/osnovni} \\
		\choice \shell{mkdir linux/osnovni} \\
		\CorrectChoice \shell{mkdir -p linux/osnovni}
	\end{oneparchoices}
	
	\question
  Our current working directory is \shell{/home/pedro/.config/hexchat/assets}. Where will we end up after running the command \shell{cd ../../././../} ? 

	\begin{oneparchoices}
		\choice \shell{/home/pedro/.config/hexchat} \\
		\choice \shell{/home/} \\
		\choice \shell{/home/pedro/.config/} \\
		\CorrectChoice \shell{/home/pedro/}
	\end{oneparchoices}
	
	\question
 Which sequence of characters redirects stdout and stderr elsewhere? 
	\begin{oneparchoices}
		\choice 2>\&1 
		\CorrectChoice \&> 
		\choice >
		\choice >2
	\end{oneparchoices}
	
	\question
\shell{yes} is a command which, in its base form, merely prints ‘y’ to the standard output. Sometimes, we don't want scripts and commands to pest us with "[y/n]" confirmations, and instead want them to simply assume ‘y’ to be the answer to all questions. The command \shell{rm -ri <directory>} is an example of such a command. Which of the following enables the behavior described above?
	
	\begin{oneparchoices}
		\choice \shell{rm -ri | yes} 
		\CorrectChoice \shell{yes | rm -ri} 
		\choice \shell{rm -ri \&\& yes} 
		\choice \shell{yes || rm -ri}
	\end{oneparchoices}
	
	\question
  In order to redirect the printing of errors (and \textbf{only} errors) encountered during the execution of some command to some file, which of the following should we append to the command being run?

	\begin{oneparchoices}
	  \choice \shell{1>./some\_file} 
		\CorrectChoice \shell{2>./some\_file} 
		\choice \shell{\&>./some\_file} 
		\choice \shell{>./some\_file}
	\end{oneparchoices}
	
	\question
	Odd man out! Which of these doesn't fit?
	
	\begin{oneparchoices}
		\CorrectChoice before
		\choice while
		\choice until
		\choice for
	\end{oneparchoices}

	\question
	The command ls -ld /etc/??? lists:
	\begin{oneparchoices}
		\choice all files whose name has at least 3 characters
		\choice all directories whose name starts with ???
    \CorrectChoice all directories whose name has exactly 3 characters
	  \choice all files named ???
	\end{oneparchoices}
	
	\question
  In bash, every variable is of the type:
	
	\begin{oneparchoices}
		\choice integer
		\CorrectChoice string
		\choice double
		\choice char
	\end{oneparchoices}
	
	\question
  In regular expressions, which character denotes End-of-Line?

	\begin{oneparchoices}
		\choice \textasciicircum
		\CorrectChoice \$
		\choice >
		\choice *
	\end{oneparchoices}

	\question
	What is the result of running \verb| sed -r "s/(.*)\.(.*)\.(.*)/\2\1\3/"| on the string A.B.C?:
	
	\begin{oneparchoices}
		\choice CBA
		\choice ABC
		\CorrectChoice BAC
		\choice CAB
	\end{oneparchoices}
	
	\question
	Choose the command which will, given a string containing any number of digits followed by a word (e.g. 1337banana, 90000mango), strip off the numbers and turn all letters to uppercase.

	\begin{oneparchoices}
		\CorrectChoice \verb|sed -r "s/([[:digit:]])*(.*)/\U\2/"| \\ 
		\choice \verb|sed -r "s/([[:digit:]])(.*)/\u\2/"|  \\
		\choice \verb|sed -r "s/([[:alnum:]])*(.*)/\u\1/"| \\
		\choice \verb|sed -r "s/([[:alnum:]])(.*)/\U\2"| 
	\end{oneparchoices}


  \question
  Which of these commands will remove the sequence of characters ".?" (without quotation marks) from the end of the line? 

		\begin{oneparchoices}
		\choice \verb|sed -r "s/(.*)(.?)$/\1/"|  \\
		\choice \verb|sed -r "s/\.*.?$/\1/"| \\
    \CorrectChoice \verb|sed -r "s/(.*)(\.\?)$/\1/"| \\ 
    \choice \verb|sed -r "s/(\.*)(\.\?)$/\2/"|
	\end{oneparchoices}

	
		\question
	The line \shell{root:T3RqrzxU1MAH3F3wtuQu/:13284:0:99999:7:::} is a representative line from which file?
	
	\begin{oneparchoices}
		\choice \shell{/etc/root} 
		\choice \shell{/etc/sudoers} 
		\choice \shell{/etc/fstab} 
		\CorrectChoice \shell{/etc/shadow}
  \end{oneparchoices}

		\question
	The line \shell{GROUPTWO    ALL = NOPASSWD: /usr/bin/updatedb, PASSWD: /bin/kill} is a representative line from which file?
	
	\begin{oneparchoices}
		\choice \shell{/etc/root} 
		\CorrectChoice \shell{/etc/sudoers} 
		\choice \shell{/etc/fstab} 
		\choice \shell{/etc/shadow}
  \end{oneparchoices}


	\question
	The primary group for each user is written in:

	\begin{oneparchoices}
   	\choice \shell{/etc/hosts}
		\CorrectChoice \shell{/etc/passwd}
		\choice \shell{/usr/passwd} 
		\choice \shell{/usr/group}  
	\end{oneparchoices}

	\question
  The command "\shell{su}" run with no arguments:
	
	\begin{oneparchoices}
		\choice gives the current user sudo privileges \\
		\CorrectChoice switches the current user with \shell{root} \\ 
		\choice creates a new user \\
		\choice returns an error due to being run improperly
	\end{oneparchoices}
	
	\question
  Which command ensures the file may only be read and nothing else? 

	\begin{oneparchoices}
		\CorrectChoice \shell{chmod 444 some\_file}\\ 
		\choice \shell{chmod 755 some\_file} \\
		\choice \shell{chmod 777 some\_file} \\
		\choice \shell{chmod 333 some\_file}
	\end{oneparchoices}
	
	\question
	Which command permits deleting files inside of a directory \textbf{only} to the directory owner and root?
	
	\begin{oneparchoices}
		\choice \shell{chmod +X some\_directory} \\
		\choice \shell{chmod +r some\_directory} \\
    	\CorrectChoice \shell{chmod +t some\_directory}\\ 	
	  \choice \shell{chmod -X some\_directory}
	\end{oneparchoices}

	\question
	Which of these is valid if you want to run a command in the background?

	\begin{oneparchoices}
		\choice \shell{<command> \$} \\
		\CorrectChoice \shell{<command> \&} \\
		\choice \shell{<command> !} \\
		\choice \shell{<command> \textasciicircum} \\
	\end{oneparchoices}
	
	\question
  What does the \shell{\$\$} variable contain?

	\begin{oneparchoices}
				\choice current kernel version\\
		\CorrectChoice \textit{PID} of the current process \\ 
		\choice \textit{exit code} of the last run program \\
		\choice the default terminal emulator
	\end{oneparchoices}
	
	\question
  Which signal is sent by pressing Ctrl+C? 

  \begin{oneparchoices}
		\choice \shell{SIGKILL}
		\choice \shell{SIGTRAP}
    \choice \shell{SIGHUP}
    \CorrectChoice \shell{SIGINT}  
  \end{oneparchoices}

	\question
  Choose the "strongest" signal:

  \begin{oneparchoices}
		\CorrectChoice \shell{SIGKILL}
		\choice \shell{SIGTRAP}
    \choice \shell{SIGHUP}
    \choice \shell{SIGINT}  
  \end{oneparchoices}


	\question
  If a user wants to suspend the active process, which of these key combinations will enable them to do it?

  \begin{oneparchoices}
		\choice \shell{CTRL+C}
		\CorrectChoice \shell{CTRL+Z}
    \choice \shell{ALT+C}
    \choice \shell{CTRL+D}  
  \end{oneparchoices}
	
\vspace{1em}
\uplevel{
	Which term do You prefer:
	\\ A. In the morning (10-12) \hspace{20mm} B. In the afternoon (12-14)
}    

\end{questions}
\end{document}
