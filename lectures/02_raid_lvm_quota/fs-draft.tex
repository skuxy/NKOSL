\section{Datotečni sustavi}
\begin{frame}
	\frametitle{Datotečni sustavi}
	\begin{itemize}
		\item Određuju način zapisivanja i čitanja podataka na mediju
	\end{itemize}
	Karakteristike datotečnih sustava
	\begin{itemize}
		\item Normiranje imena datoteka i upravljanje direktorijima
		\item Metadata na datotekama
		\item Upravljanje prostorom na mediju
		\item \dots
	\end{itemize}
\end{frame}

\begin{frame}
	\frametitle{Neki datotečni sustavi}
	ext - \emph{extended filesystem}
	\begin{itemize}
		\item Osmišljen za Linux sustave
		\item ext2 najkorišteniji
		\item Struktura metadata iz Unix file systema (permissions, timestamps)
		\item Datoteke predstavljene strukturom \emph{inode}
		\item ext3 i ext4 - Nadogradnje na ext2
		\item ext3 uvodi \emph{journal} - struktura file systema se čuva u izvanrednim okolnostima
	\end{itemize}
	FAT (vfat)\
	\begin{itemize}
		\item Masovna podrška
		\item Najveća veličina datoteke 4 GiB (FAT16)
		\item exFAT - naprednija verzija
	\end{itemize}
\end{frame}

\begin{frame}
	\frametitle{swap i tmpfs}
	\begin{itemize}
		\item Posebne strukture u Linuxu
	\end{itemize}
	swap
	\begin{itemize}
		\item \emph{Paging} particija
		\item Dio virtualne memorije
	\end{itemize}
	tmpfs
	\begin{itemize}
		\item Spremanje podataka na RAM
	\end{itemize}
\end{frame}


\begin{frame}[fragile]
	\frametitle{\texttt{/etc/fstab}}
\begin{verbatim}
# <filesys>  <dir>  <type>  <options>          <dump> <pass>
/dev/sda1    /      ext4    defaults,noatime   0      1
/dev/sda2    none   swap    defaults           0      0
/dev/sdb1    /home  ext4    defaults,noatime   0      2
tmpfs        /tmp   tmpfs   nodev,nosuid       0      0
\end{verbatim}
\end{frame}

\section{Struktura direktorija}
\begin{frame}
	\frametitle{UNIX directory structure}
	\begin{itemize}
		\item Hijerarhijski raspored datoteka dostupnih na UNIX sustavu
	\end{itemize}
	\vfill
	\emph{Everything is a file}
	\vfill
	\begin{itemize}
		\item Diskovne particije se \emph{montiraju (mount)} na lokacije u sustavu
		\item Raspored direktorija određen je Filesystem Hierarchy Standardom
	\end{itemize}
	\vfill
\end{frame}

\begin{frame}
	\frametitle{FHS struktura}
	\texttt{/} - root\\
	\begin{tabular}{p{3cm} l}
		\texttt{bin} & \emph{Osnovne} korisničke izvršne datoteke \\
		\texttt{boot} & Datoteke bootloadera \\
		\texttt{dev} & Device datoteke \\
		\texttt{etc} & Konfiguracija sustava \\
		\texttt{home} & Matični direktoriji korisnika \\
		\texttt{lib} & Biblioteke i kernel moduli \\
		\texttt{opt} & Razni softver \\
		\texttt{root} & Matični direktorij korisnika \texttt{root} \\
		\texttt{sbin} & Sistemske izvršne datoteke \\
		\texttt{srv} & Podaci servisa na računalu \\
		\texttt{tmp} & Privremeni podaci \\
		\texttt{usr} & Dijeljeni dio strukture \\
		\texttt{var} & Često mijenjani i privremeni podaci 
	\end{tabular}
\end{frame}