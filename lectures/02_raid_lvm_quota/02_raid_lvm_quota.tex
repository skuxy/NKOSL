\documentclass[t]{beamer}

\usetheme{CambridgeUS}
\usecolortheme{beaver}
\setbeamertemplate{navigation symbols}{}

\usepackage[utf8]{inputenc}
\usepackage[croatian]{babel}

\usepackage{datetime}
\renewcommand{\dateseparator}{.}
\newcommand{\todayiso}{\twodigit\day \dateseparator \twodigit\month \dateseparator \the \year}
\date{\todayiso}

\usepackage{listing}
\usepackage{graphicx}
\usepackage{subcaption}
\captionsetup{compatibility=false}

\title[NKOSL]{Napredno korištenje operacijskog sustava Linux}
\author[Dominik Barbarić]{Dominik Barbarić\\{\small Nositelj: doc.dr.sc. Stjepan Groš}}
\subtitle{2. RAID, LVM i kvote}
\institute[FER]{Sveučilište u Zagrebu\\Fakultet elektrotehnike i računarstva}

\begin{document}

{
	\setbeamertemplate{footline}{}
	\begin{frame}
		\maketitle
	\end{frame}
}

\begin{frame}
	\frametitle{Sadržaj}
	\tableofcontents
\end{frame}

\begin{frame}
	\frametitle{Uobičajene diskovne arhitekture}
	Diskovi se u sustavu prikazuju kao logičke jedinice:
	{\ttfamily
		\begin{itemize}
			\item[] /dev/sda
			\item[] /dev/sdb
			\item[] /dev/hda
		\end{itemize}
	}
	\textbf{Problem:} Pronaći metode za efikasno upravljanje raspoloživim diskovnim prostorom
	\vfill
	\begin{itemize}
		\item \textbf{JBOD} - Just a bunch of disks
		\begin{itemize}
			\item Diskovi se koriste neovisno
		\end{itemize}
		\item \textbf{Spanned}
		\begin{itemize}
			\item Prazan prostor na više diskova se koristi kao jedan logički disk
		\end{itemize}
	\end{itemize}
\end{frame}

\section{RAID}
\begin{frame}
	\frametitle{RAID}
	\begin{itemize}
		\item \textbf{RAID} - Redundant array of independent disks
	\end{itemize}
	\begin{itemize}
		\item \emph{RAID polje} - logička jedinica sastavljena od više fizičkih diskova
		\item Prednosti
		\begin{itemize}
			\item Povećanje prostora
			\item Povećanje performansi
			\item Redundancija (zaštita) podataka
		\end{itemize}
	\end{itemize}
	\begin{itemize}
		\item RAID-om se upravlja
		\begin{description}
			\item[Sklopovski] RAID kontrolerom
			\item[Softverski] md
		\end{description}
	\end{itemize}
	\begin{itemize}
		\item \emph{RAID level} - Način rada RAID polja
	\end{itemize}
\end{frame}

\begin{frame}
	\frametitle{RAID kontroler}
	\framesubtitle{RAID ROM}
	\includegraphics[width=\textwidth]{Intel_RAID.png}
\end{frame}

\subsection{Osnovni RAID leveli}
\begin{frame}
	\frametitle{RAID 0}
	
	\begin{columns}[T]
	\begin{column}{0.6\textwidth}
		\textbf{Striping}
		\begin{itemize}
			\item Podaci se raspodjeljuju na više diskova
		\end{itemize}
		\begin{itemize}
			\item Povećanje prostora
			\item Povećanje performansi
			\item Nema zaštite podataka
		\end{itemize}
	\end{column}
	\begin{column}{0.3\textwidth}
		\includegraphics[width=\textwidth]{200px-RAID_0.png}
	\end{column}
	\end{columns}
\end{frame}

\begin{frame}
	\frametitle{RAID 1}
	
	\begin{columns}[T]
	\begin{column}{0.6\textwidth}
		\textbf{Mirror}
		\begin{itemize}
			\item Podaci se kopiraju na više diskova
		\end{itemize}
		\begin{itemize}
			\item Zaštita podataka
			\item Nema povećanja prostora ni performansi
		\end{itemize}
	\end{column}
	\begin{column}{0.3\textwidth}
		\includegraphics[width=\textwidth]{200px-RAID_1.png}
	\end{column}
	\end{columns}
\end{frame}

\begin{frame}
	\frametitle{RAID 5}
	
	\begin{columns}[T]
	\begin{column}{0.55\textwidth}
		\textbf{Block-striping with distributed parity}
		\begin{itemize}
			\item Podaci se raspodjeljuju na više diskova
			\item Svakom bloku podataka se izračunava paritet i zapisuje na jedan od diskova
		\end{itemize}
		\begin{itemize}
			\item Povećanje prostora
			\begin{itemize}
				\item Potrebno osigurati dodatni prostor za paritet
			\end{itemize}
			\item Povećanje performansi
			\item Zaštita podataka
		\end{itemize}
	\end{column}
	\begin{column}{0.45\textwidth}
		\includegraphics[width=\textwidth]{500px-RAID_5.png}
	\end{column}
	\end{columns}
\end{frame}

\begin{frame}
	\frametitle{Ostali RAID leveli}
	\emph{Nisu u (širokoj) upotrebi}\\
	\vfill
	\textbf{RAID 2}
	\begin{itemize}
		\item Hammingov kod za zaštitu podataka
		\item Dedicirani hard diskovi za zaštitne bitove
	\end{itemize}
	\vfill
	\textbf{RAID 3, 4}
	\begin{itemize}
		\item Paritetna zaštita
		\item Dedicirani hard disk za paritetne bitove
	\end{itemize}
	\vfill
	\textbf{RAID 6}
	\begin{itemize}
		\item Distribuirani zaštitni blokovi
		\item Dvostruki paritetni blokovi
	\end{itemize}
\end{frame}

\subsection{Ugniježđeni RAID leveli}
\begin{frame}
	\frametitle{RAID 0+1}
	
	\begin{columns}[T]
	\begin{column}{0.55\textwidth}
		\textbf{Stripe, then mirror}
		\begin{itemize}
			\item Podaci se raspodjeljuju unutar jednog polja pa se cijelo polje kopira
		\end{itemize}
		\begin{itemize}
			\item Prednosti RAID 0 na razini jednog polja
			\item Sigurnost RAID 0 polja
		\end{itemize}
	\end{column}
	\begin{column}{0.45\textwidth}
		\includegraphics[width=\textwidth]{500px-RAID_01.png}
	\end{column}
	\end{columns}
\end{frame}

\begin{frame}
	\frametitle{RAID 1+0}
	
	\begin{columns}[T]
	\begin{column}{0.55\textwidth}
		\textbf{Mirror, then stripe}
		\begin{itemize}
			\item Podaci se kopiraju unutar jednog polja pa se cijelo polje raspodjeljuje
		\end{itemize}
		\begin{itemize}
			\item Sigurnost RAID 1 na razini jednog polja
		\end{itemize}
	\end{column}
	\begin{column}{0.45\textwidth}
		\includegraphics[width=\textwidth]{500px-RAID_10.png}
	\end{column}
	\end{columns}
\end{frame}

\subsection{Softverski RAID}
\begin{frame}
	\frametitle{Softverski RAID}
	\textbf{md} - multiple device
	\begin{itemize}
		\item Linux implementacija softverskog RAID-a
		\item Podržava
		\item[] Span, RAID 0, RAID 1, RAID 4, RAID 5, RAID 6, Nested
	\end{itemize}

	\texttt{mdadm}
	\begin{itemize}
		\item[] \texttt{/dev/md*}
		\item[] Particionirana polja
		\begin{itemize}
			\item[] \texttt{/dev/md/md1p1}
			\item[] \texttt{/dev/md/md2p1}
			\item[] \dots
		\end{itemize}
	\end{itemize}
	
	\texttt{/proc/mdstat} - popis inicijaliziranih polja\\
	\begin{itemize}
		\item[] \texttt{mdadm} ne pamti polja pri ponovnom pokretanju
		\item[$\rightarrow$] \texttt{mdadm --detail --scan >> /etc/mdadm.conf}
	\end{itemize}
\end{frame}

\begin{frame}
	\frametitle{RAID boot}
	\textbf{Hardverski RAID}
	\begin{itemize}
		\item[] OS vidi RAID polja kao logičke diskove. Ne može pristupiti fizičkim diskovima.
		\item[$\rightarrow$] Bootloader radi kao u konfiguraciji bez RAID-a.
	\end{itemize}
	\vfill
	\textbf{Softverski RAID}
	\begin{itemize}
		\item[] OS vidi fizičke diskove i iz njih gradi logičke.
		\item[$\rightarrow$] Bootloader mora imati podršku za takva polja.
	\end{itemize}
	\vfill
	\textbf{Fake RAID}
	\begin{itemize}
		\item[] Hibridni model. Kontroler ima ograničenu RAID podršku.
		\item[$\rightarrow$] Bootloader vidi RAID polja kako logičke diskove. Ovisno o hardveru može bootati i bez dodatnih modula.
	\end{itemize}
\end{frame}

\section{LVM}
\begin{frame}
	\frametitle{LVM}
	\textbf{Logical volume manager}
	\begin{itemize}
		\item Fleksibilnije upravljanje diskovnim prostorom
		\item Implementacija kroz \textbf{device mapper} (dm)
	\end{itemize}
	\begin{itemize}
		\item Moguće dodavanje, uklanjanje i zamjena diskova za vrijeme rada sustava (čak i bez unmount-a)
	\end{itemize}
\end{frame}

\begin{frame}
	\frametitle{LVM arhitektura}
	\textbf{Physical volume} (PV)
	\begin{itemize}
		\item Particije na fizičkim diskovima
		\item Koriste se particije tipa \emph{Linux LVM}
		\item LVM ih dijeli na manje jedinice - \textbf{Physical extent} (PE)
	\end{itemize}
	\textbf{Logical volumes} (LV)
	\begin{itemize}
		\item Logički disk (particija)
		\item LVM ih dijeli da manje jedinice - \textbf{Logical extent} (LE)
	\end{itemize}
	\vfill
	\textbf{Volume group} (LV)
	\begin{itemize}
		\item Grupira više PV i LV u jednu skupinu radi mogućnosti upravljanja
	\end{itemize}
	\textbf{}
\end{frame}

\begin{frame}
	\frametitle{LVM arhitektura}
	\includegraphics[width=\textwidth]{lvm_arch.png}
\end{frame}

\begin{frame}[fragile]
	\frametitle{LVM}
	\framesubtitle{Primjer}
	\textbf{Kreiranje LVM logičke particije korištenjem dviju fizičkih particija}\\
	\begin{verbatim}
pvcreate /dev/sda1 /dev/sdb2
vgcreate vg-name /dev/sda1 /dev/sdb2

# Informacije o VG
vgscan
vgdisplay vg-name

lvcreate -l 100%FREE -n lvm0 vg-name
mkfs -t ext3 /dev/lvm-disk/lvm0
	\end{verbatim}
\end{frame}

\section{Loop devices}
\begin{frame}[fragile]
	\frametitle{Loop devices}
	\begin{itemize}
		\item Interpretacija običnih datoteka kao uređaja
		\item Datoteci se dodjeljuje \emph{loop} uređaj u \texttt{/dev} folderu kojem se pristupa kao običnom disku
	\end{itemize}
	\begin{itemize}
		\item Datoteka može sadržavati datotečni sustav
	\end{itemize}
	\vfill
	\textbf{Primjer stvaranja loop device-a}:
	\begin{verbatim}
# Prazna 100MiB datoteka
dd if=/dev/zero of=device.img bs=512 count=2048

losetup /dev/loop0 device.img
mkfs -t ext3 /dev/loop0
mount -t ext3 /dev/loop0 /mnt/image
	\end{verbatim}
\end{frame}

\section{Kvote}
\begin{frame}
	\frametitle{Kvote}
	\begin{itemize}
		\item Ograničavaju korištenje diskovnog prostora
	\end{itemize}
	\begin{description}
		\item[usrquota] Korisničke kvote
		\item[grpquota] Grupne kvote
	\end{description}
	\begin{itemize}
		\item \emph{Obične} kvote
		\item \emph{Journaled} kvote
		\begin{itemize}
			\item Vode zapise o promjenama na disku što povećava pouzdanost
		\end{itemize}
	\end{itemize}
	\vspace{1em}
	{\ttfamily \small
		quotacheck: Your kernel probably supports journaled quota but you are not using it.
		Consider switching to journaled quota to avoid running quotacheck after an unclean
		shutdown.
	}
\end{frame}

\begin{frame}[fragile]
	\frametitle{Kvote}
	\framesubtitle{Podešavanje i naredbe}
	Datoteka \texttt{/etc/fstab}
	{\small \begin{verbatim}
# Obicne kvote
/dev/sda2  /home  ext4  defaults,usrquota,grpquota   0  1
# Journaled kvote
/dev/sda2  /home  ext4  defaults,usrjquota=aquota.user,
                 grpjquota=aquota.group,jqfmt=vfsv0  1  1
	\end{verbatim} }
	U prvom direktoriju trebaju biti datoteke \texttt{aquota.user} i \texttt{aquota.group}
	\begin{itemize}
		\item[] \texttt{/home/aquota.user}
		\item[] \texttt{/home/aquota.group}
	\end{itemize}
	\begin{verbatim}
quotacheck -avgum
quotaon -avgu
	\end{verbatim}
\end{frame}

\begin{frame}[fragile]
	\frametitle{Kvote}
	\framesubtitle{Podešavanje i naredbe}
	{\footnotesize \begin{verbatim}
# repquota -a

*** Report for user quotas on device /dev/md0
Block grace time : 7 days ; Inode grace time : 7 days
                         Block limits                 File limits
User          used       soft     hard  grace   used  soft hard  grace
-----------------------------------------------------------------------
root    --       52         0        0            10     0    0
veljko  -- 25585028  40000000 40000000          1123     0    0
cetko   --  5162460  40000000 40000000            49     0    0
marin   --  6498572  10000000 20000000           183     0    0
deni    --  5903852  10000000 20000000           528     0    0
lovro   --  3649796  10000000 20000000            19     0    0
matej   +- 11334792  10000000 20000000  2 days   646     0    0
	\end{verbatim}}
\end{frame}

\begin{frame}[fragile]
	\frametitle{Kvote}
	\framesubtitle{Podešavanje i naredbe}
	\begin{description}
		\item[Soft limit] Aktivacija \emph{grace period-a} za vrijeme korisnik još može koristiti prostor
		\item[Hard limit] Limit nakon kojeg korisnik nema mogućnost pisanja po disku
	\end{description}
	\vspace{1em}
	{\footnotesize \begin{verbatim}
# edquota cetko

Disk quotas for user cetko (uid 1001):
 Filesystem         blocks        soft        hard inodes soft hard
 /dev/md0          5162460    40000000    40000000     49    0    0
	\end{verbatim}}
\end{frame}

\section*{}
\begin{frame}
	\frametitle{Literatura}
	\texttt{man mdadm}
	\url{http://www.ducea.com/2009/03/08/mdadm-cheat-sheet/}
	\vfill
	\url{http://debian-handbook.info/browse/wheezy/advanced-administration.html}\\
	\url{https://wiki.archlinux.org/index.php/Software_RAID_and_LVM}\\
	\url{http://www.tuxradar.com/content/lvm-made-easy}
	\vfill
	\texttt{man losetup}
	\vfill
	\url{https://wiki.archlinux.org/index.php/disk_quota}
\end{frame}

\end{document}
