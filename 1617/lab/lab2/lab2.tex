\documentclass[12pt,a4paper]{article}

\usepackage[croatian]{babel}
\usepackage[utf8]{inputenc}

\usepackage[margin=2cm]{geometry}
\usepackage[colorlinks=true,urlcolor=black]{hyperref}
\pagenumbering{gobble}

\begin{document}
	\title{Laboratorijska vježba 2}
	\date{\vspace{-5ex} 29.03.2017.}
	\maketitle

Zadatke je potrebno napraviti proizvoljno na vlastitom računalu, virtualnom stroju ili serveru. Preporučamo distribucije bazirane na Debianu.

\section*{1. Zadatak - mreža}

Saznali ste da će zbog radova na žičnoj mreži uskoro prestati raditi DHCP server.
Kako bi izbjegli potencijalan gubitak veze s internetom u to vrijeme,
odlučili ste postaviti \textbf{statičku IP adresu} svom mrežnom uređaju.
Ne želite uzrokovati probleme drugim korisnicima mreže pa ste mu odlučili dodijeliti
istu IP adresu koja vam je automatski dodjeljena od DHCP servera pri spajanju na mrežu.
Također za svaki slučaj odlučujete koristiti Google javni DNS server dostupan na adresi \textbf{8.8.8.8} i \textbf{8.8.4.4} \\

\underline{\textbf{Potrebno je:}}
\begin{itemize}
	\item konfigurirati \textit{/etc/network/interfaces} tako da mrežni uređaj koristi statičku IP adresu
  \item eksplicitno osim IP adrese navesti: subnet masku, adresu gateway uređaja, adrese primarnog i sekundarnog DNS servera
  \item kopirati konfiguracijsku datoteku u svoj home direktorij za laganu demonstraciju
\end{itemize}


\newpage


\section*{2. Zadatak - systemd}

Vaš kolega je razvio vrlo impresivnu web aplikaciju čiji je izvorni kod dostupan na: \\ \url{https://pastebin.com/KwGiKByi}. \\
Ona sluša na portu 1337 i s obzirom što upišete u URL, aplikacija odgovara. \\
Recimo, ako aplikaciji pristupite uz URL \textit{http://localhost:1337/hello-world}, ona odgovori odgovarajućom porukom.
Aplikacija također ispisuje odgovarajuću informaciju na \textit{stdout}. \\
\\
Iako vrsan developer, kolega je tek počeo proučavati Linux pa traži vašu pomoć.
Do sada je aplikaciju pokretao iz naredbenog retka pomoću \textit{python webserver.py},
međutim ne zna kako da drži svoju aplikaciju uključenu izvan terminala, kako da se aplikacija pali pri pokretanu računala. \\
Sram ga je priznati ali ima i bug koji ne zna popraviti. Ako koji korisnik slučajno posjeti \textit{/crash}, aplikacija mu crasha! \\
Htio bi privremeno, dok ne popravi bug, da se aplikacija automatski restarta u slučaju crasha. \\
Kolega je čuo da je systemd dobar alat za ovaj posao i traži vašu pomoć. \\
Razmišljajući veselo na putu do vas o svojoj aplikaciji, sjetio se da je inače mogao pratiti ponašanje korisnika u svojoj aplikaciji direktno iz terminala.
Zabrinulo ga je to na trenutak, ali se sjetio da systemd ima ugrađen logging sustav - journal \\
Zamolio vas je da se pobrinete i da može kroz journalctl promatrati uporabu svoje aplikacije. \\

\underline{\textbf{Potrebno je:}}
\begin{itemize}
	\item napraviti systemd unit (service) file koji pali danu python skriptu
	\item pobrinuti se da se aplikacija automatski restarta u slučaju crasha
	\item postaviti unit tako da se pali pri pokretanju računala
  \item osigurati da se servis diže nakon što je dignuta mreža
	\item pratiti korištenje aplikacije (logove) kroz journalctl
\end{itemize}


\newpage


\section*{3. Zadatak - nginx}

Ambiciozni kolega iz prošlog zadatka je odlučio podići svoj sustav na novu razinu dodajući u njega nginx.
Neki od korisnika su mu javili da im se ne sviđa to što moraju upisivati port 1337 na kraju adrese kako bi pristupili aplikaciji.
Kolega zna da browseri implicitno koriste port 80 i odlučio je prebaciti svoju aplikaciju baš na taj port! Ali nije sve tako lako.
Uz spomenutu aplikaciju, kolega želi kroz browser posluživati i neke datoteke, isto na portu 80. \\
Ne sviđa mu se ideja dodavanja logike za posluživanje datoteka u svoju aplikaciju jer to nije njena poanta.
Dodatno, kolegica mu je na kavi rekla da python nije baš najbrži jezik i da bi takvo posluživanje datoteka moglo biti sporo.\\
Zato je kolega odlučio dodati fleksibilnost i brzinu u svoj sustav u obliku nginx web servera. I tražiti vašu pomoć. \\
\\
Zamislio je to ovako... Na njegovoj adresi će na portu 80 slušati nginx
i nginx će s obzirom na putanju http zahtjeva raditi različite stvari. \\
\\
Ako putanja počinje sa \textit{/app}, nginx će proslijediti zahtjev aplikaciji
koju ste mu pomogli pretvoriti u servis u drugom zadatku na portu 1337 \\
\\
Ako putanja počinje sa \textit{/public}, nginx će sam poslužiti datoteke iz datotečnog sustava na lokaciji \textit{/home/USERNAME/public/}.
Npr. ako je napravljen zahtjev \textit{http://localhost/public/slika.jpg}, nginx treba, ako postoji, odgovoriti s datotekom \textit{/home/USERNAME/public/slika.jpg} \\
\\
Oduševljen zamišljenom fleksibilnošću, kolega se odlučuje malo zabaviti i podesiti nginx
tako da na \textit{/kset} odgovara sa privremenim http redirectom na \textit{https://www.kset.org}. \\
\\
Kolega nije zaboravio na važnost logova, pa planira informacije o posjetima spremati na lokaciju
\textit{/home/USER/public/nginx-access.log} jer je ljubitelj transparentnog poslovanja,
a informacije o greškama iz sigurnosnih razloga na \textit{/var/log/nginx/nginx-error.log} \\
\\

Napomene:
\begin{itemize}
	\item treba vam samo jedan nginx config file, recimo... /etc/nginx/sites-available/localhost
  \item /etc/nginx/sites-enabled/localhost je obično simlink na /etc/nginx/sites-available/localhost
  \item nakon promjene konfiguracijskih datoteka, potrebno ih je ponovno učitati pomoću \textit{nginx -s reload}
  \item treba vam samo jedan nginx server block
  \item pripazite na ime servera u konfiguraciji - inače se koristi domena koja se poslužuje, vama sada to zamjenjuje "localhost"
  \item privremeni http redirect ima kod 302
  \item u praksi je gotovo uvijek sigurnosni propust javno izlagati logove, obično se i access log i error log drže u /var/log
  \item \textbf{BONUS} zadatak: konfigurirati nginx da sve poslužuje preko https protokola, uz self signed certifikat i redirect sa http na https
\end{itemize}


\end{document}
