\documentclass[12pt,a4paper]{article}

\usepackage[croatian]{babel}
\usepackage[utf8]{inputenc}

\usepackage[margin=2cm]{geometry}
\usepackage[colorlinks=true,urlcolor=black]{hyperref}
\pagenumbering{gobble}

\begin{document}
	\title{Laboratorijska vježba 3}
	\date{\vspace{-5ex} 06.05.2017.}
	\maketitle

Zadatke je potrebno napraviti proizvoljno na vlastitom računalu, virtualnom stroju ili serveru. Izbor operacijskog sustava prepuštamo vama - ako se osjećate pustolovno, zadatke možete pokušati riješiti i na Windowsima.

\section*{1. Zadatak}

Učeći za netom protekle međuispite, listali ste po Redditu, te ste primjetili mnogo članaka koji hvale ili grde \textit{Docker} tehnologiju. Da barem malo ublažite grižnju savjesti zbog toga što visite na Redditu, odlučili ste i sami vidjeti o čemu se radi. Za prvi zadatak s tom tehnologijom odlučili ste uzeti nešto banalno: kontejner koji neprekidno skuplja podatke o trenutnom stanju vremena u Zagrebu.

\underline{\textbf{Potrebno je:}}
\begin{itemize}
  \item instalirati \textit{Docker} i \textit{docker-compose} na željeno računalo
  \item napisati \textit{Dockerfile} za servis koji bi sa nekog API-ja dohvaćao podatke o trenutnom vremenskom stanju u Zagrebu svakih nekoliko sekundi i ispisivao ih na običan stdout
  
  \item primjer naredbe: \textit{curl -s wttr.in/Zagreb} $\vert$ \textit{head -7}
  
  \item nakon što ste se uvjerili da servis propisno radi, napisati \textit{docker-compose} koji bi stvorio 4 instance tog servisa i pokrenite ga (nemojte naknadno zaboraviti ugasiti te servise)
\end{itemize}
  Razmislite - kako bi se mogao pogledati stdout pojedinog kontejnera?


\newpage


\section*{2. Zadatak}

Marljivo i dalje učeći na Redditu, naišli ste na \textit{r/dataisbeautiful}\footnote{\url{https://www.reddit.com/r/dataisbeautiful}}, \textit{subreddit} koji glorificira vizualizaciju različitih podataka, te ste poželjeli spremiti te rezultate na lokalno računalo kako biste ih kasnije mogli parsirati u neki graf ili tablicu, s konačnim ciljem stjecanja vječne slave na \textit{r/dataisbeautiful}.

Kako biste to napravili, potrebno je modificirati postojeće \textit{Dockerfileove} da se u kontejnere \textit{mounta} lokalni \textit{volume}, te da ti servisi rezultate pišu u njih.


\underline{\textbf{Potrebno je:}}
\begin{itemize}
	\item stvoriti novi direktorij i \textit{mountati} ga na kontejnere
	\item preusmjeriti ispis servisa da zapisuje i na \textit{stdout} i u \textit{fileove} u novom direktoriju 
	\item BONUS: Proučite kako biste u zasebnom kontejneru stvorili bazu podataka i \textit{linkali} ju s postojećim kontejnerima
\end{itemize}



\end{document}
