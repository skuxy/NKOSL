\documentclass[12pt,a4paper]{article}

\usepackage[croatian]{babel}
\usepackage[utf8]{inputenc}

\usepackage[margin=2cm]{geometry}
\usepackage[colorlinks=true,urlcolor=black]{hyperref}
\pagenumbering{gobble}

\begin{document}
	\title{Laboratorijska vježba 3}
	\date{\vspace{-5ex} 06.05.2017.}
	\maketitle

Zadatke je potrebno napraviti proizvoljno na vlastitom računalu, virtualnom stroju ili serveru. Operacijski sustav ne bi trebao biti bitan, no ako se osjećate pustolovno zadatke možete pokušati riješiti i na Windowsima.

\section*{1. Zadatak}

Učeći za ove međuispite listali ste po Redditu, te ste primjetili mnogo članaka koji hvale ili grde Docker tehnologiju, pa ste ju odlučili i sami vidjeti o čemu se radi. Za (vjerojatno) prvi zadatak s tom tehnologijom odlučili ste uzeti nešto banalno: kontejner koji neprekidno skuplja podatke o trenutnom stanju vremena u Zagrebu.

\underline{\textbf{Potrebno je:}}
\begin{itemize}
  \item instalirati \textit{Docker} i \textit{docker-compose} na željeno računalo
  \item napisati Dockerfile za servis koji bi sa nekog API-ja dohvaćao podatke o trenutnom vremenskom stanju u Zagrebu svakih nekoliko sekundi i ispisivao ih na običan stdout
  
  \item primjer naredbe: \textit{curl -s wttr.in/Zagreb} \vert$ \textit{head -7}
  
  \item Nakon što ste se uvjerili da servis propisno radi, napišite docker-compose koji bi stvorio 4 instance tog servisa i pokrenite ga. Nemojte naknadno zaboraviti ugasiti te servise.
  \item Razmislite kako bi se mogao pogledati stdout pojedinog kontejnera?
\end{itemize}


\newpage


\section*{2. Zadatak}

Marljivo i dalje učeći na Redditu, naišli ste na r/dataisbeautiful, te ste poželjeli spremiti te rezultate na lokalno računalo kako bi ih kasnije mogli parsirati u neki graf ili tablicu, kako bi sakupili mnogo imaginarnih internet bodova. 

Kako bi to napravili, potrebno je modificirati postojeće Dockerfileove da se u kontejnere mounta lokalni volume, te da ti servisi rezultate pišu u njih.


\underline{\textbf{Potrebno je:}}
\begin{itemize}
	\item stvoriti novi direktorij i mountati ga na containere
	\item preusmjeriti ispis servisa da zapisuje i u \textit{fileove} u novom direktoriju, i na stdout
	\item BONUS: Proučite kako bi u zasebnog kontejneru stvorili bazu podataka \itemit{linkali} ju s postojećim kontejnerima
\end{itemize}



\end{document}
