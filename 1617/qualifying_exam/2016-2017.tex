\documentclass[a4paper,11pt]{exam}
\usepackage[left=1.5cm, right=1.5cm, top=3cm]{geometry}
\usepackage[utf8]{inputenc}
\usepackage[T1]{fontenc}
\setlength\parindent{0pt}
\renewcommand{\familydefault}{\sfdefault}
\newcommand{\shell}[1]{\texttt{#1}}

\CorrectChoiceEmphasis{\bfseries\itshape}
\printanswers

\usepackage{xpatch}
\xpatchcmd{\oneparchoices}{\penalty -50\hskip 1em plus 1em\relax}{\hfill}{}{}
\xpatchcmd{\oneparchoices}{\penalty -50\hskip 1em plus 1em\relax}{\hfill}{}{}

\begin{document}
\firstpageheader
	{}
	{{\large \textbf{Napredno korištenje OS Linux}}\\
		\textbf{Prijemni ispit - 1. ožujka 2017.}}
	{}
\footer
	{}{}{\thepage}
	
Ime i prezime: \fillin[][7cm] \hfill JMBAG: \fillin[][5cm]\\

\begin{questions}
	\question
	Gruba podjela operacijskih sustava je na:
	
	\begin{oneparchoices}
		\CorrectChoice Windows i UNIX bazirane
		\choice Windows i Linux bazirane
		\choice Linux i Unix bazirane
		\choice Linux i iOS bazirane
	\end{oneparchoices}
	
	\question
	Debian, Fedora i OpenSUSE samo su neke od Linux
	
	\begin{oneparchoices}
		\CorrectChoice distribucija
		\choice klasa
		\choice dijelova
		\choice segmenata
	\end{oneparchoices}
	
	\question
	Terminiranje naredbe moguće je korištenjem kratice na tipkovnici:
	
	\begin{oneparchoices}
		\choice Ctrl + X
		\choice Ctrl + V
		\CorrectChoice Ctrl + C
		\choice Ctrl + S
	\end{oneparchoices}
	
	\question
    Ispis: \shell{lrwxrwxrwx 1 root root 4 2010-04-29 10:44 /bin/sh -> bash} \\može se dobiti korištenjem naredbe:
    
	\begin{oneparchoices}
		\CorrectChoice \shell{ls -l /bin/sh}
        \choice \shell{dir -p /bin/sh}
        \choice \shell{grep -c /bin/sh}
        \choice \shell{cat –all /bin/sh}
	\end{oneparchoices}

	\question
    U prvom bloku ispisa iz prethodnog zadatka, tri su skupine niza znakova "rwx". Druga po redu skupina predstavlja dozvole:
	
	\begin{oneparchoices}
        \choice vlasnika datoteke
        \CorrectChoice grupe vlasnika datoteke
        \choice ostalih korisnika
        \choice svih korisnika
	\end{oneparchoices}
	
	\question
	Trenutno se nalazite u direktoriju \shell{/home/korisnik/Dokumenti/faks/diglog}. Kojom od ponuđenih naredbi ćete se pozicionirati u \shell{/home/korisnik/Desktop}?
	
	\begin{oneparchoices}
		\choice \shell{Cd .././../../Desktop} \\
		\choice \shell{cd ././.././Desktop} \\
		\CorrectChoice \shell{cd .././../../Desktop} \\
		\choice \shell{cd ../../../../Desktop}
	\end{oneparchoices}
	
	\question	
	Za preimenovanje datoteka možemo koristiti naredbu:
	
	\begin{oneparchoices}
		\choice rn
		\CorrectChoice mv
		\choice cp
		\choice rm
	\end{oneparchoices}
	
	\question
	Kojom od sljedećih naredbi stvaramo direktorij "linux" i u njemu direktorij "osnovni"? 
	
	\begin{oneparchoices}
		\choice \shell{touch -p linux/osnovni} \\
		\choice \shell{touch linux/osnovni} \\
		\choice \shell{mkdir linux/osnovni} \\
		\CorrectChoice \shell{mkdir -p linux/osnovni}
	\end{oneparchoices}
	
	\question
	Što ne vrijedi za naredbu \shell{touch}?
	
	\begin{oneparchoices}
		\choice njome možemo stvoriti novu praznu datoteku \\
		\CorrectChoice njome možemo promijeniti sva MAC vremena odjednom, ali ne i pojedinačno \\
		\choice njome možemo mijenjati MAC vremena direktorija \\
		\choice njome možemo mijenjati MAC vremena datoteke
	\end{oneparchoices}
	
	\question
	Da bismo preusmjerili stdout ispis neke naredbe u datoteku /tmp/ispis, na kraj naredbe dodajemo: 
	
	\begin{oneparchoices}
		\choice \shell{0>/tmp/ispis} 
		\CorrectChoice \shell{1>/tmp/ispis} 
		\choice \shell{2>/tmp/ispis} 
		\choice \shell{3>/tmp/ispis}
	\end{oneparchoices}
	
	\question
	Naredbom \shell{wc}, kojoj je kao argument zadana datoteka, u općenitom slučaju ne možemo ispisati:
	
	\begin{oneparchoices}
		\choice broj riječi u datoteci
		\CorrectChoice broj slova u datoteci
		\choice broj znakova u datoteci
		\choice broj linija u datoteci
	\end{oneparchoices}
	
	\question
	Kojim znakom označavamo početak reda u regularnim izrazima? 
	
	\begin{oneparchoices}
		\CorrectChoice \textasciicircum
		\choice '
		\choice <
		\choice \$ 
	\end{oneparchoices}
	
	\question
	Izbaci uljeza!
	
	\begin{oneparchoices}
		\CorrectChoice leave
		\choice Ctrl+D
		\choice logout
		\choice exit
	\end{oneparchoices}

	\question
	Redak \shell{root:x:0:0:root:/root:/bin/bash} primjer je retka iz datoteke:
	
	\begin{oneparchoices}
		\choice \shell{/etc/hosts} 
		\choice \shell{/etc/sudoers} 
		\choice \shell{/etc/fstab} 
		\CorrectChoice \shell{/etc/passwd}
	\end{oneparchoices}
	
	\question
	U retku \shell{root:x:0:0:root:/root:/bin/bash}, što označava druga nula?
	
	\begin{oneparchoices}
		\choice ID korisnika
		\CorrectChoice ID primarne grupe
		\choice broj prijava na sustav
		\choice broj znakova u lozinki
	\end{oneparchoices}
	
	\question
	Za (trajnu) promjenu korisnika prijavljenog unutar ljuske možemo koristiti naredbu:
	
	\begin{oneparchoices}
		\choice \shell{sudo}
		\CorrectChoice \shell{su}
		\choice \shell{switchuser}
		\choice \shell{who}
	\end{oneparchoices}
	
	\question
	Što ne vrijedi za naredbu \shell{adduser}?
	
	\begin{oneparchoices}
		\choice može stvoriti novu grupu \\
		\CorrectChoice može dodati više korisnika odjednom \\
		\choice može dodati postojećeg korisnika u postojeću grupu \\
		\choice može stvoriti korisnika bez matičnog direktorija
	\end{oneparchoices}
	
	\question
	Direktorij \shell{/etc/skel} sadrži:
	
	\begin{oneparchoices}
		\CorrectChoice standardne konfiguracijske datoteke za korisnika \\
		\choice obično je prazan \\
		\choice slike kostura raznih životinja \\
		\choice skripte koje sustav izvršava pri pokretanju
	\end{oneparchoices}

	\question
	Kako bismo korisniku vlasniku datoteke, bez mijenjanja ostalih dozvola, omogućili pisanje u tu datoteku?
	
	\begin{oneparchoices}
		\CorrectChoice \shell{chmod u+w datoteka} \\
		\choice \shell{usermod +w datoteka} \\
		\choice \shell{chmod u=w datoteka} \\
		\choice \shell{chmod 200 datoteka} \\
	\end{oneparchoices}

	\question
	Ako želimo osigurati da se skripta koju smo napisali pokrene u bash ljusci, moramo u prvi red skripte napisati:
	
	\begin{oneparchoices}
		\choice \shell{\$!/bin/bash} \\ 
		\choice \shell{!\#/bin/bash} \\ 
		\CorrectChoice \shell{\#!/bin/bash} \\ 
		\choice \shell{./bin/bash}
	\end{oneparchoices}
	
	\question
	Izbaci uljeza!
	
	\begin{oneparchoices}
		\CorrectChoice \shell{SIGSTOP} 
		\choice \shell{SIGINT} 
		\choice \shell{SIGTERM} 
		\choice \shell{SIGKILL}
	\end{oneparchoices}
	
	\question
	Matični direktorij \shell{root} korisnika nalazi se:
	
	\begin{oneparchoices}
		\choice u /sys direktoriju
		\choice u /home direktoriju
		\CorrectChoice u korijenskom (root) direktoriju
		\choice nema svoj matični direktorij
	\end{oneparchoices}
	
	\question
	Izvorni proces koji je proces-roditelj svim procesima naziva se:
	
	\begin{oneparchoices}
		\CorrectChoice \shell{init} 
		\choice \shell{root}  
		\choice \shell{parent} 
		\choice \shell{start}
	\end{oneparchoices}
	
	\question
	Izbaci uljeza!
	
	\begin{oneparchoices}
		\choice ifconfig
		\choice iwconfig
		\choice ip
		\CorrectChoice ipconfig
	\end{oneparchoices}


\vspace{1em}
\uplevel{
	U idućim zadacima potrebno je odrediti ispravni regularni izraz koji zadovoljava uvjete zadatka, koji se prosljeđuje naredbi egrep.
}


	\question
	Iz datoteke ispisati sve retke u kojima se \textbf{ne} nalaze praznine:
	
	\begin{oneparchoices}
		\choice \shell{egrep "[[:blank:]]" datoteka} \\
		\CorrectChoice \shell{egrep -v "[[:blank:]]" datoteka} \\
		\choice \shell{egrep -c "[[:blank:]]" datoteka} \\
		\choice \shell{egrep "[\textasciicircum[:blank:]]" datoteka} \\
	\end{oneparchoices}
	
	\question
	Ispisati količinu redaka datoteke u kojima se nalazi znakovni niz "nux" na kraju retka: 
	
	\begin{oneparchoices}
		\choice \shell{egrep "nux\$" datoteka | wc -w} \\
		\CorrectChoice \shell{egrep -c "nux\$" datoteka} \\
		\choice \shell{egrep -c "nux\%" datoteka | wc -l} \\
		\choice \shell{egrep "nux\%" datoteka|} 
	\end{oneparchoices}
	
	\question
	Iz datoteke ispisati sve retke u kojima se nalazi niz "C++":
	
	\begin{oneparchoices}
		\choice \verb|egrep "C+{2}" datoteka| \\
		\choice \verb|egrep "C++" datoteka| \\
		\CorrectChoice \verb|egrep "C\+{2}" datoteka| \\
		\choice \verb|egrep "C+\+" datoteka|
	\end{oneparchoices}
	
	\question
	Iz datoteke ispisati sve retke u kojima se veliko slovo nalazi između dva mala slova (npr "mAn"): 
	
	\begin{oneparchoices}
		\choice \verb|egrep "[[:lower:]]?[A-Z]{1}[[:lower:]]?" datoteka| \\
		\choice \verb|egrep "[^A-Z][A-Z][^A-Z]" datoteka| \\
		\choice \verb|egrep "[a-z][[:alpha:]][a-z]" datoteka| \\
		\CorrectChoice \verb|egrep "[a-z][[:upper:]][a-z]" datoteka|
	\end{oneparchoices}
	
    
\vspace{1em}
\uplevel{
	U idućim zadacima potrebno je odrediti ispravni izraz koji zadovoljava uvjete zadatka, koji se prosljeđuje naredbi sed -r.
}


	\question Prva i zadnja tri znaka retka zamijene mjesta (npr.: \shell{kineski zid} $\longrightarrow$ \shell{zideski kin}): 
	
	\begin{oneparchoices}
		\choice \verb|'s/(...).*(...)/\2\1/'| \\
		\choice \verb|'s/(...)(.*)(...)/\2\3\1/'| \\
		\CorrectChoice \verb|'s/(...)(.*)(...)/\3\2\1/'| \\
		\choice \verb|'s/(..)(.*)(..)/\3\2\1/'| 
	\end{oneparchoices}
	
	\question Ako je zarez na kraju retka, obriši taj zarez (npr.: \shell{dakle,} $\longrightarrow$ \shell{dakle}): 

	\begin{oneparchoices}
		\CorrectChoice \verb|'s/(.*)\,$/\1/'| \\
		\choice \verb|'s/(.*)\,/\1/'| \\
		\choice \verb|'s/.*(\,)/\1/'| \\
		\choice \verb|'s/(.*)\,$/\U\1/'|
	\end{oneparchoices}

	\vspace{1em}
\uplevel{
	Preferencija termina:
	\\ A. Jutarnji (10-12) \hspace{20mm} B. Popodnevni (12-14)
}
	%\begin{oneparchoices}
	%	\choice Jutarnji (10-12) \\
	%	\choice Popodnevni (12-14)
	%\end{oneparchoices}

\end{questions}
\end{document}
