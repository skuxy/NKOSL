\documentclass[12pt,a4paper]{article}

\usepackage[croatian]{babel}
\usepackage[utf8]{inputenc}

\usepackage[margin=2cm]{geometry}
\usepackage[colorlinks=true,urlcolor=black]{hyperref}
\pagenumbering{gobble}

\begin{document}
	\title{Domaća zadaća 3}
	\date{\vspace{-5ex} 15.05.2015.}
	\maketitle
	Evan Stone se sve češće susreće s mrežnim problemima. Želi znati točno vrijeme kada oni nastupaju kako bi lakše utvrdio uzrok problema. Pomognite Evanu da logira informacije koje bi mu pomogle.\\
	\section*{Zadatak}
	Potrebno je napisati skriptu \texttt{ping-log.sh} koja će svakih 10 sekundi pokrenuti naredbu\\
	\hspace*{3em} \texttt{ping 8.8.8.8}\\
	da provjeri vrijeme trajanja odgovora.\\
	Svaki ping reply treba logirati u datoteku \texttt{/var/log/nkosl/ping.log}. Na početku svakog retka mora pisati timestamp kada je naredba pokrenuta.\\
	\hspace*{3em} Primjer: \texttt{[2015-05-15 13:37:00+02:00]}\\
	Za navedenu log datoteku potrebno je napisati logrotate konfiguraciju koja će jednom dnevno arhivirati log. Arhivirani log u nazivu mora imati datum arhiviranja.
	\newline
	\newline
	Kod pokretanja skripte \texttt{ping-log.sh}, također je potrebno zapisati koliko je opterećenje procesora (CPU load) te stanje memorije i particija (ukupno, slobodno, zauzeto).
	\newline
	\newline
	Napisati skriptu \texttt{tcpdump-log.sh} koja će na mrežnom uređaju pratiti i bilježiti pakete koje šalje ping naredba u datoteku \texttt{/var/log/nkosl/tcpdump.log}. Pratiti samo pakete ping naredbe (ICMP protokol) i to pakete čija je odredišna adresa ona kojoj šaljemo ICMP paket našom skriptom. Nije potrebno konfigurirati logrotate. 	
\end{document}