\documentclass[12pt,a4paper]{article}

\usepackage[croatian]{babel}
\usepackage[utf8]{inputenc}

\usepackage[margin=2cm]{geometry}
\usepackage[colorlinks=true,urlcolor=black]{hyperref}
\pagenumbering{gobble}

\begin{document}
	\title{Domaća zadaća 1}
	\date{\vspace{-5ex} 10.04.2016.}
	\maketitle
    Domaća zadaća sastoji se od 3 zadatka. Potrebno je rješenja svih zadataka zapakirati u tar.gz arhivu i poslati na NKOSL mailing listu.

	\section{Zadatak}
	Potrebno je napisati dijelove konfiguracije za program Syslog, tj. dijelove \texttt{/etc/syslog.conf} datoteke.
	\begin{itemize}
        \item Zapisivanje isključivo \texttt{warning} poruka u datoteku /var/log/messages.warning
        \item Zapisivanje \texttt{user error} poruka na /dev/tty9
        \item Zapisivanje svih \texttt{emergency} poruka na 192.168.0.200.
        \item Prokomentirajte: na kojem portu Syslog daemon slusa vanjske log poruke?
        \item Prokomentirajte: kako omogućiti prihvaćanje log poruka s drugog računala?
    \end{itemize}

    \section{Zadatak}
    Potrebno je napisati logrotate konfiguraciju za imaginarni program Program.
    Program tijekom svog izvođenja zapisuje log poruke u \texttt{/var/log/program/} direktorij.
    Kao rješenje priložite konfiguraciju i navedite punu putanju do nje. \\
    Za konfiguraciju treba uzeti u obzir slijedeće pretpostavke:
    \begin{itemize}
        \item Pregled logova po danima 
        \item Program generira puno zapisa unutar jednog dana
        \item Potrebno je čuvati 2 godine logova
        \item Memorijski prostor je ograničen
        \item Samo Root korisnik i grupa Firma smiju čitati i pisati u log datoteku
        \item Program otpušta log file descriptor primitkom HUP signala
    \end{itemize}
        
    \section{Zadatak}
    
    Za potrebe upoznavanja s programima za nadzor računala i rješavanje sljedećeg zadatka, potrebno je instalirati program Zabbix.
    Konfigurirajte Zabbix agent da prati iskorištenost vaših procesora kroz vrijeme.
    Kao rješenje priložite slike grafova iz Zabbix programa.
	
\end{document}
