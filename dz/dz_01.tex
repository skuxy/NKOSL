\documentclass[12pt,a4paper]{article}

\usepackage[croatian]{babel}
\usepackage[utf8]{inputenc}

\usepackage[margin=2cm]{geometry}
\usepackage[colorlinks=true,urlcolor=black]{hyperref}
\pagenumbering{gobble}

\begin{document}
	\title{Domaća zadaća 2}
	\date{\vspace{-5ex} 22.03.2015.}
	\maketitle
	Bryan Matthew se zaposlio na praksi u tvrtci Sevilla Ltd. Njihov sistem administrator je shvatio da Bryan provodi puno vremena na chatu i njegovoj IP adresi zabranio pristup IRC protokolu. Međutim, Bryan je zadnji vikend puno toga naučio o mrežama i shvatio da može napraviti virtualni interface \texttt{eth0:0} kojeg će moći koristiti za IRC.
	\section*{Zadatak}
	Na sučelju \texttt{eth0} je potrebno napraviti novi virtual interface i konfigurirati ga na sljedeći način:
	\begin{itemize}
		\item Sučelje pripada mreži 172.16.92.64/26
		\item Osigurajte pristup mreži 172.16.0.0/16 preko novog interfacea i routera na adresi 172.16.92.120
		\item Sučelje odbacuje sve dolazne pakete osim onih na TCP portu 6667 (za IRC protokol)
	\end{itemize}
	Konfiguracija mora biti trajna - nakon reboota sučelje i dalje mora ispravno raditi.\\
	{\small Ovisno o distribuciji, moguće da se fizičko mrežno sučelje ne zove \texttt{eth0}. Koristite nazive koje vrijede na vašem sustavu.}\\
	\,\\
	Kao rješenje domaće zadaće šaljete sljedeće datoteke:
	\begin{itemize}
		\item Bash skriptu kojom ste ostvarili traženu konfiguraciju
		\item \texttt{ifconfig.log} - datoteka koja sadrži ispis naredbe \texttt{ifconfig}
		\item \texttt{route.log} - datoteka koja sadrži ispis naredbe \texttt{route -n}
		\item \texttt{iptables.log} - datoteka koja sadrži ispis naredbe \texttt{iptables -nvL}
		\item[] {\small Umjesto predloženih naredbi možete koristiti bilo koje druge naredbe iz koje je vidljiva konfiguracija sučelja. Promijenite ime datoteke tako da odgovara korištenoj naredbi.}
	\end{itemize}
	Tražene datoteke upakirajte u .tar arhivu.
	
\end{document}