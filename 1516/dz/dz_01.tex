\documentclass[12pt,a4paper]{article}

\usepackage[croatian]{babel}
\usepackage[utf8]{inputenc}

\usepackage[margin=2cm]{geometry}
\usepackage[colorlinks=true,urlcolor=black]{hyperref}
\pagenumbering{gobble}

\renewcommand*{\familydefault}{\sfdefault}
\renewcommand*{\sfdefault}{lmss}

\begin{document}
	\title{Domaća zadaća 1}
	\date{\vspace{-5ex} 11.04.2016.}
	\maketitle

	\section{Zadatak}
	Potrebno je napisati dijelove \texttt{syslog} konfiguracije, tj. dijelove \texttt{/etc/syslog.conf} datoteke tako da se ostvare sljedeće funkcionalnosti:
	\begin{itemize}
        \item Sve \textit{warning} poruke se zapisuju u datoteku \texttt{/var/log/messages.warning}.
        \item Sve \textit{user error} poruke se ispisuju na terminal \texttt{/dev/tty9}.
        \item Sve \textit{emergency} poruke se prosljeđuju \texttt{syslog} serveru na adresi \texttt{192.168.0.200}.
    \end{itemize}
    U konfiguracijskoj datoteci u komentarima odgovorite na pitanja:
    \begin{enumerate}
        \item Na kojem portu \texttt{syslog} daemon sluša vanjske log poruke?
        \item Kako omogućiti prihvaćanje log poruka s drugog računala?
    \end{enumerate}

    \section{Zadatak}
    Potrebno je napisati \texttt{logrotate} konfiguraciju za program čija se izvršna datoteka zove \texttt{user1}.
    Program tijekom svog izvođenja zapisuje log poruke u \texttt{/var/log/user1/} direktorij.
    Kao rješenje priložite tar arhivu u kojoj se nalazi konfiguracijska datoteka s punom putanjom do nje.\\
    (\textit{Dakle, primjer strukture tar arhive je} \verb|var -> conf -> file.conf|)\\
    
    \noindent Konfiguracija treba ostvariti sljedeće funkcionalnosti:
    \begin{itemize}
        \item Logovi moraju biti odvojeni po danima.
        \item Logovi se rotiraju prilikom dostizanja limita veličine datoteke. Limit uzmite proizvoljno.
        \item Logovi se čuvaju 2 godine.
        \item Svi rotirani logovi moraju biti kompresirani.
        \item Samo \texttt{root} korisnik i grupa \texttt{firma} smiju čitati i pisati u log datoteku.
    \end{itemize}
    Prilikom rotiranja logova imati na umu da program \texttt{user1} otpušta log file descriptor primitkom \texttt{SIGHUP} signala.
        
    \section{Zadatak}
    
    Za potrebe upoznavanja s programima za nadzor računala i rješavanje sljedećeg zadatka potrebno je instalirati program \texttt{Zabbix}.
    Konfigurirajte Zabbix agent da prati iskorištenost vaših procesora kroz vrijeme.
    Kao rješenje priložite slike proizvoljnih grafova iz Zabbix programa.
	
\end{document}
