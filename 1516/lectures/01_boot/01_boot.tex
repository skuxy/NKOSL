\documentclass[t]{beamer}

\usetheme{CambridgeUS}
\usecolortheme{beaver}
\setbeamertemplate{navigation symbols}{}

\usepackage[utf8]{inputenc}
\usepackage[croatian]{babel}

\usepackage{datetime}
\renewcommand{\dateseparator}{.}
\newcommand{\todayiso}{\twodigit\day \dateseparator \twodigit\month \dateseparator \the \year}
\date{\todayiso}

\usepackage{listing}
\usepackage{graphicx}
\usepackage{subcaption}
\captionsetup{compatibility=false}


\title[NKOSL]{Napredno korištenje operacijskog sustava Linux}
\author[Dominik Barbarić, Josip Domšić]{Dominik Barbarić, Josip Domšić\\{\small Nositelj: Stjepan Groš}}
\subtitle{1. Boot proces i datotečni sustavi}
\institute[FER]{Sveučilište u Zagrebu\\Fakultet elektrotehnike i računarstva}

\begin{document}

{
	\setbeamertemplate{footline}{}
	\begin{frame}
		\maketitle
	\end{frame}
}

\begin{frame}
	\frametitle{Sadržaj}
	\tableofcontents
\end{frame}

\section{Boot}
\begin{frame}
    \frametitle{Od uključivanja do /bin/bash}
    \begin{itemize}
        \item BIOS 
        \item MBR ili UEFI
        \item Bootloader i odabir OS-a
        \item Učitavanje initram i kernela
        \item Init proces 
        \item Bash
    \end{itemize}
\end{frame}


\section{BIOS}
\begin{frame}
	\frametitle{BIOS}
	\textbf{B}asic \textbf{I}nput and \textbf{O}utput \textbf{S}ystem
	\begin{itemize}
		\item Firmware na matičnoj ploči
		\item Pokreće se na startupu
		\item Izvršava Power-on self-test (POST) nad matičnom i drugim spojenim uređajima (S.M.A.R.T. test diskova, Extension ROM dodatnih uređaja itd.)
		\item Konfigurira se preko ugrađenog firmwarea ili hardverski (jumperi, DIP switchevi, ... )
		\item Omogućuje odabir uređaja s kojeg se pokreće operacijski sustav
	\end{itemize}
\end{frame}

\begin{frame}
	\frametitle{Particije}
	\begin{itemize}
		\item Hard disk sadrži particije formatirane određenim datotečnim sustavom
		\item Informacije o particijama sadržane u
		\begin{itemize}
			\item Master boot record (MBR)
			\item GUID Partition Table (GPT)
		\end{itemize}
	\end{itemize}
\end{frame}	

\begin{frame}
	\frametitle{MBR}
	\begin{itemize}
		\item Zapisan u prvom sektoru diska (512 B / 2 KiB)
		\begin{itemize}
			\item Partition table
			\item \textbf{Bootstrap code} - minimalni podaci potrebni za nastavak boot procesa
			\item ...
		\end{itemize}
		\item Podržava 4 \emph{primarne} particije
		\item Za veći broj particija koriste se \emph{logičke} particije koje se upisuju u jednu primarnu particiju tipa \emph{extended}
	\end{itemize}
\end{frame}

\section{Bootloader}
\begin{frame}
	\frametitle{Bootloader}
	\begin{itemize}
		\item Izvršava se nakon POST
		\item Pokreće operacijski sustav - učitava kernel
	\end{itemize}
	
	\begin{itemize}
		\item Prva faza bootloadera zapisana u bootstrap code MBR diskova
		\item Često se pokreće druga faza s neke od particija
	\end{itemize}
	\begin{itemize}
		\item GPT diskovi
		\begin{itemize}
	 		\item Drugi pristup bootloaderu na UEFI sustavima
		\end{itemize}
	\end{itemize}
\end{frame}

\begin{frame}
	\frametitle{Bootloderi}
	\textbf{GR}and \textbf{U}nified \textbf{B}ootloader
	\begin{itemize}
		\item GNU projekt
		\item Konfigurabilan za više operacijskih sustava na jednom računalu
	\end{itemize}
	\begin{itemize}
		\item \textbf{GRUB Legacy}
		\begin{itemize}
			\item Starija verzija
			\item Više se ne razvija
		\end{itemize}
		\item \textbf{GRUB 2}
		\begin{itemize}
			\item Drugačije konfiguracijske datoteke
			\item Modularan, puno mogućnosti
			\item Zauzima puno više prostora od GRUB Legacy
		\end{itemize}
	\end{itemize}
	\vfill
	\textbf{Syslinux}
	\begin{itemize}
		\item \emph{Lightweight} bootloader
		\item Jednostavna konfiguracija
	\end{itemize}
\end{frame}

\begin{frame}
	\frametitle{Druge platforme}
	Linux pokreće bootloader koji je često ugrađen u firmware platforme
	\begin{figure}
		\centering
		\begin{subfigure}[f]{0.55\textwidth}
			\includegraphics[width=\textwidth]{ps2kernelloader.jpg}
			\caption*{Play Station 2 \emph{kernelloader}}
		\end{subfigure}
		\hfill
		\begin{subfigure}[f]{0.3\textwidth}
			\includegraphics[width=\textwidth]{androidHboot.jpg}
			\caption*{HTC Android Hboot}
		\end{subfigure}
	\end{figure}
\end{frame}

\begin{frame}
	\frametitle{UEFI}
	\textbf{U}nified \textbf{E}xtensible \textbf{F}irmware \textbf{I}nterface
	\begin{itemize}
		\item Specifikacija sučelja između OS-a i firmware-a
		\item Razvijen 2005 iz Intelovog EFI sustava
		\item Zamjena za BIOS
		\item Servisi, aplikacije, driveri ...
		\begin{itemize}
			\item EFI system partition
		\end{itemize}
	\end{itemize}
	\vfill
	\begin{itemize}
		\item \textbf{GUID partition table} (GPT)
		\begin{itemize}
			\item Najveći doprinos specifikacije
			\item Podržavaju ga i mnogi BIOS-i
			\item Legacy MBR / Protective MBR omogućuje ograničeni \textit{backward compatibility} s MBR (BIOS sustavima)
		\end{itemize}
	\end{itemize}
\end{frame}

\begin{frame}
	\frametitle{UEFI}
	\begin{itemize}
		\item Boot manager je dio UEFI specifikacije
		\item Traži i izvršava bootloader sa neke od particija
		\item Detekcija UEFI bootloadera na sustavu
	\end{itemize}
	\begin{itemize}
		\item \textit{Compatibility support module (CSM)}
		\begin{itemize}
			\item Omogućuje \textit{backward compatibility} s klasičnim bootloaderima
		\end{itemize}
	\end{itemize}
\end{frame}

\begin{frame}
	\frametitle{initrd / initramfs}
	\begin{itemize}
		\item Kernel prepoznaje sve uređaje na sustavu i učitava
		\item[] \emph{initial ramdisk}
		\item[]
		\item Omogućuje kernelu učitavanje osnovnih modula kako bi došao do \emph{root} filesystema
		\begin{itemize}
			\item moduli
			\item enkripcija
			\item RAID / LVM
			\item \dots
		\end{itemize}
		\item[]
		\item initrd se po završetku briše iz memorije
		\item Pokreće se init
	\end{itemize}
\end{frame}

\section{init, systemd}
\begin{frame}
	\frametitle{init}
	\begin{itemize}
		\item System V uvodi koncept \emph{runlevel}-a
	\end{itemize}
	\begin{itemize}
		\item U Linuxu definirano nekoliko runlevel-a
		\begin{description}
			\item[0] - Sustav se isključuje
			\item[1 / S] - Single user mode
			\item[2 - 5] - Multi user mode (ili nešto drugo)
			\item[6] - Ponovno pokretanje
		\end{description}
		\item Stvarni \textit{startup} level je određen u datoteci \texttt{/etc/inittab}
	\end{itemize}
\end{frame}

\begin{frame}
	\frametitle{init}
	\begin{itemize}
		\item Kod ulaska u svaki runlevel pokreću se skripte
		\item[] \texttt{/etc/rc.d/rc?.d}
		\item[]
		\item Poredak izvođenja skripti je određen imenom skripte u folderu:
		\begin{itemize}
			\item[] \textbf{K01-K99} - skripta gasi servis pri ulasku u runlevel
			\item[] \textbf{S01-S99} - skripta pokreće servis pri ulasku u runlevel
		\end{itemize}
		\item Nakon izvođenja svih skripti u folderu pokreće se i \texttt{/etc/rc.local}
		\item[]
	\end{itemize}
	Sve što se nalazi u direktorijima \texttt{/etc/rc.d} je obično samo symlink na skripte u \texttt{/etc/init.d}
\end{frame}

\begin{frame}
	\frametitle{systemd}
	\begin{itemize}
		\item Zamjenjuje init na sve više distribucija
		\item Sustav upravljanja procesima i cijelim sustavom
	\end{itemize}
	\begin{itemize}
		\item Konfiguracija zasnovana na \emph{unit} datotekama
		\item Neki tipovi \emph{unit} datoteka
		\begin{description}
			\item[service] opisuje servis na sustavu
			\item[target] grupira unit-e u skupine
			\item[\dots]
		\end{description}
	\end{itemize}
	\vfill
	\texttt{systemctl} \\ \texttt{journalctl}
\end{frame}

\section{Kernel}
\begin{frame}[fragile]
	\frametitle{Kernel moduli}
	Kernel modules are pieces of code that can be loaded and unloaded into the kernel upon demand. They extend the functionality of the kernel without the need to reboot the system.
	\begin{itemize}
		\item Moduli su spremljeni na lokaciji
		\item[] \verb|/usr/lib/modules/kernel_release|
	\end{itemize}
	\begin{itemize}
		\item \texttt{lsmod} - trenutno učitani moduli
		\item \texttt{modprobe module\_name parameter\_name=parameter\_value} - učitavanje modula
		\item \texttt{modprobe -r module\_name} - \textit{unload} modula
	\end{itemize}
	\begin{itemize}
		\item[] \verb|/etc/modprobe.d/|
		\item[] \verb|  options module_name parameter_name=parameter_value|
	\end{itemize}
\end{frame}
\begin{frame}[fragile]
	\frametitle{Kernel domesticus}
	\framesubtitle{Kompajliranje kernela (CentOS)}
    \footnotesize
    \begin{verbatim}
	yum -y groupinstall 'Development tools'
	yum -y install ncurses-devel rpm-build qt-devel
	\end{verbatim}
	\begin{verbatim}
	cd /usr/src/kernels
	wget https://www.kernel.org/pub/linux/kernel/v3.x/linux-3.12.6.tar.bz2
	tar xvjf linux-3.12.6.tar.bz2
	cd /usr/src/kernels/linux-3.12.6
	\end{verbatim}
	Izmijeniti konfiguraciju (promjeniti verziju u Makefile):
	\begin{verbatim}
	cp /boot/config`-uname -r` .config
	make menuconfig # ili make xconfig
	\end{verbatim}
	Spremiti .config i kopajlirati
	\begin{verbatim}
	make; make modules_install; make install;
	vi /boot/grub/grub.conf # (default=1 to default=0)
	\end{verbatim}
\end{frame}

\section*{}
\begin{frame}
	\frametitle{Literatura}
	\url{https://wiki.archlinux.org/index.php/GRUB_Legacy}\\
	\url{https://wiki.archlinux.org/index.php/GRUB}\\
	\url{http://wiki.gentoo.org/wiki/GRUB2_Quick_Start}\\
	\url{https://wiki.archlinux.org/index.php/Syslinux}
	\vfill
	\url{https://wiki.archlinux.org/index.php/Kernel_parameters}
	\url{https://wiki.archlinux.org/index.php/Unified_Extensible_Firmware_Interface}
	\vfill
	\url{http://users.cecs.anu.edu.au/~okeefe/p2b/power2bash/power2bash.html}\\
\end{frame}

\end{document}
